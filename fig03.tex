\begin{figure*}
\begin{center}
\[
\begin{array}{ccc}
%\multicolumn{3}{c}{\includegraphics*[width=150mm]{Figures/Fig3.png}} \\
\mbox{$E = 1.0 \times 10^{-3}$, $Pm = 5.0$} &
\mbox{$E = 1.0 \times 10^{-4}$, $Pm = 2.0$} \\
\multicolumn{2}{c}{\mbox{$r_{i} / r_{o} = 0.15$}} \\
\includegraphics*[width=75mm]{Figures/Ene_vs_RaRatio_015_E3.pdf} &
\includegraphics*[width=75mm]{Figures/Ene_vs_RaRatio_015_E4.pdf} \\
\multicolumn{2}{c}{\mbox{$r_{i} / r_{o} = 0.25$}} \\
\includegraphics*[width=75mm]{Figures/Ene_vs_RaRatio_025_E3.pdf} &
\includegraphics*[width=75mm]{Figures/Ene_vs_RaRatio_025_E4.pdf} \\
\multicolumn{2}{c}{\mbox{$r_{i} / r_{o} = 0.35$}} \\
\includegraphics*[width=75mm]{Figures/Ene_vs_RaRatio_035_E3.pdf} &
\includegraphics*[width=75mm]{Figures/Ene_vs_RaRatio_035_E4.pdf}
\end{array}
\]
\end{center}
% \caption{The kinetic and magnetic energy density as a function of the ratio of Rayleigh number to the critical Rayleigh number $Ra / Ra_{\rm crit}$ in spherical shells with different geometries. The black, red, and blue points are the $E_{\rm kin}$ values in the non-MHD cases, $E_{\rm kin}$ values in the MHD cases, and $E_{\rm mag}$ values in the MHD cases, respectively. The magnetic energy for failed dynamo cases is plotted in the shaded area. Results for the regimes with $E = 1.0 \times 10^{-3}$, $Pm = 5.0$ are shown in the left columns, and results for the regimes with $E = 1.0 \times 10^{-4}$, $Pm = 2.0$ are shown in the right columns.
\caption{The kinetic and magnetic energy densities as a function of the ratio of the Rayleigh number to the critical Rayleigh number, $Ra / Ra_{\rm crit}$, in spherical shells with different geometries.
Results for $E = 10^{-3}$ and $Pm = 5.0$ are shown in the left column, and those for $E = 10^{-4}$ and $Pm = 2.0$ in the right column.
The black, red, and blue symbols are the $E_{\rm kin}$ values in the non-MHD cases, the $E_{\rm kin}$ values in the MHD cases, and the $E_{\rm mag}$ values in the MHD cases, respectively.
The magnetic energy density for failed dynamo cases is plotted in the shaded area. 
}
\label{fig:fig_3}
\end{figure*}