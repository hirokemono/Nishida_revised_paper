\section{METHOD}

A numerical geodynamo model is given by an electrically conducting Boussinesq fluid in a rotating spherical shell. The governing equations of the geodynamo in the outer core are described by the momentum equation, continuity equation, heat equation, magnetic induction equation, and Gauss's law for the magnetic field, which are respectively given as
%
\begin{equation}
\begin{array}{l}
\displaystyle
\frac{\partial \bvec{u}}{\partial t} + \left(\bvec{\omega} \times \bvec{u}\right)
 = - \nabla \left(P+\frac{1}{2}u^{2} \right) - \nabla \times \nabla \times \bvec{u}
\\
\displaystyle
\hspace*{3em}
      - \frac{2}{E} \left(\hat{\bvec{z}} \times \bvec{u} \right)
  + \frac{Ra}{Pr} T \frac{\bvec{r}}{r_{o}}
        + \frac{1}{Pm E} \left(\bvec{J} \times \bvec{B} \right),
\label{eq:momentum}
\end{array}
\end{equation}
%
\begin{equation}
\nabla \cdot \bvec{u} = 0, 
\label{eq:conservation}
\end{equation}
%
\begin{equation}
\frac{\partial T}{\partial t} + \left(\bvec{u} \cdot \nabla \right) T
 = \frac{1}{Pr} \nabla^{2} T,
\label{eq:heat}
\end{equation}
%
\begin{equation}
 \frac{\partial \bvec{B}}{\partial t}
 = -\frac{1}{Pm}  \nabla \times \nabla \times \bvec{B}
       + \nabla \times \left(\bvec{u} \times \bvec{B} \right),
\label{eq:induction}
\end{equation}
%
and
\begin{equation}
\nabla \cdot \bvec{B} = 0,
\label{eq:Gauss_B}
\end{equation}
%
where $\bvec{u}$, $t$, $P$, $T$, $\bvec{B}$, $\bvec{r}$, and $\hat{\bvec{z}}$ are the velocity, time, reduced pressure, temperature, magnetic field, position vector, and unit vector along the rotation axis, respectively.
In eq.~(\ref{eq:momentum}), $\bvec{\omega} = \nabla \times \bvec{u}$ is the vorticity, and $\bvec{J} = \nabla \times \bvec{B} / \mu_0$ in the dimensional form is the current density, where $\mu_0$ is the magnetic permeability of vacuum.
In eqs.~(\ref{eq:momentum})--(\ref{eq:Gauss_B}), the length and temperature are normalised by the outer core thickness, $L = r_{o} - r_{i}$ and average temperature difference between the inner core boundary (ICB) and the CMB, $\Delta T$, respectively. 
The time is normalised by the kinematic viscosity diffusion time, $\tau_{\nu}  = L^{2} / \nu$, where $\nu$ is the kinetic viscosity, and the magnetic field is normalised by $\sqrt{\rho_{0} \mu_{0} \eta \Omega}$, where $\rho_{0}$, $\eta$, and $\Omega$ are the average density of core fluid, the magnetic diffusivity, and the angular velocity of system's rotation, respectively.
The Rayleigh number, $Ra$, the Ekman number, $E$, the Prandtl number, $Pr$, and the magnetic Prandtl number, $Pm$, are respectively defined as follows:
%
\begin{equation}
Ra = \displaystyle{ \frac{\alpha g_o \Delta T L^{3}}{ \kappa \nu} }, 
E  = \displaystyle{ \frac{\nu}{\Omega L^{2}} },
Pr = \displaystyle{ \frac{\nu}{\kappa} }, 
\mbox{ and }
Pm = \displaystyle{ \frac{\nu}{\eta} },
\label{eq:dimensionless}
\end{equation}
%
where, $\alpha$, $g_o$, and $\kappa$ are the thermal expansion coefficient, the gravitational acceleration at the CMB, and the thermal diffusivity, respectively.

We use a numerical dynamo code Calypso \cite{Matsui:2014}, in which numerical simulations are carried out in the spherical coordinates, $(r, \theta, \phi)$.
Solenoidal vector fields, $\bvec{u}$ and $\bvec{B}$, are decomposed into the toroidal and poloidal constituents as
%{\color{red}
%
\begin{equation}
\bvec{u}(\bvec{r}, t) = \nabla \times (u_T (\bvec{r}, t) \hat{\bvec{r}} ) + \nabla \times \nabla \times (u_S (\bvec{r}, t) \hat{\bvec{r}} ) ,
\label{eq:u_uT_uS}
\end{equation}
%
\begin{equation}
\bvec{B}(\bvec{r}, t) = \nabla \times (B_T (\bvec{r}, t) \hat{\bvec{r}} ) + \nabla \times \nabla \times (B_S (\bvec{r}, t) \hat{\bvec{r}} ) ,
\label{eq:B_BT_BS}
\end{equation}
%
where $\hat{\bvec{r}}$ is the radial unit vector.
Todoidal and poloidal scalar functions for the velocity and magnetic fields, $u_T(\bvec{r}, t)$, $u_S(\bvec{r}, t)$, $B_T(\bvec{r}, t)$, and $B_S(\bvec{r}, t)$ are expanded into spherical harmonics.
As an example, $u_T (\bvec{r}, t)$ is expanded as
%
\begin{equation}
u_T (\bvec{r}, t) = \sum_{l = 1}^{L_{\rm max}} \sum_{m=-l}^{l} u_{T l}^{\ m} (r, t) Y_l^{|m|} (\theta, \phi) ,
\label{eq:uT_Ylm}
\end{equation}
%
where $L_{\rm max}$ is the truncation of spherical harmonics, and
%
\begin{equation}
Y_l^{|m|} (\theta, \phi) = \left\{
 \begin{array}{ll}
 P_l^m(\cos\theta)\cos m\phi & (m = 0, 1, 2, \cdots, l)
 \\
 P_l^{|m|}(\cos\theta)\sin |m|\phi & (m = -1, -2, \cdots, -l) .
 \end{array}
\right.
\label{eq:def_of_Ylm}
\end{equation}
%
$P_l^m(\cos\theta)$ is a Schmidt-normalised associated Legendre polynomial with degree $l$ and order $m$.
The temperature, $T(\bvec{r}, t)$, is also expanded into spherical harmonics as
%
\begin{equation}
T (\bvec{r}, t) = \sum_{l = 0}^{L_{\rm max}} \sum_{m=-l}^{l} T_l^m(r, t) Y_l^{|m|} (\theta, \phi) .
\label{eq:T_Ylm}
\end{equation}
%
%}

The radial grid points are defined as
%
\begin{equation}
r_n = r_i + \frac{r_o - r_i}{2} \left\{ 1 - \cos \left( \pi \frac{n-1}{N_r-1} \right) \right\} ~~\;\;\;\; (n = 1, \cdots , N_r) ,
\label{eq:def_of_rn}
\end{equation}
%
where $N_r$ is the number of radial grids, and spatial resolution near the boundary surfaces is expected to be high enough.
The second-order finite differences are used for the partial differentiation in the radial direction.
For the time integration, the Crank-Nicolson method is used in the diffusion and other linear terms, and the second order Adams-Bashforth method is used in the other terms.

As the initial condition, temperature perturbation is applied to all the sectorial modes $(l = m)$. 
The initial magnetic field is set as an axial dipole field and zonal toroidal field with degree $l=2$ adopted by Christensen {\it et al.} \shortcite{Uli:2001}. 
For the boundary condition, the non-dimensional temperatures at the CMB and ICB are fixed as $T(r_{o}) = 0$ and  $T(r_{i}) = 1$, respectively. 
The mantle and inner core are assumed to be co-rotating, and a non-slip boundary ($\bvec{u} = \bvec{0}$) is applied to the CMB and ICB. 
The mantle and inner core are also assumed to be electrically insulated, and the magnetic field at the boundaries is connected to the potential field.
%{\color{blue}
We choose two parameter settings; % for the Ekman, Parndtl, and magnetic Prandtl numbers. 
in one setting, the Ekman, Prandtl, and magnetic Prandtl numbers are fixed at $E = 10^{-3}$, $Pr = 1$, and $Pm = 5$, and in the other, $E = 10^{-4}$, $Pr = 1$, and $Pm = 2$ are chosen. 
To investigate the effects of different inner core sizes, the ratios of the inner core radius to the outer core radius are set as $r_{i} / r_{o} = 0.15$, $0.25$ and $0.35$ for each parameter setting, and $Ra$ is also changed in each setting. 
Parameters and spatial resolution, $L_{\rm max}$ and $N_{r}$, are listed in Tables \ref{table:Summary_15} to 8 
% \ref{table:SummaryEk4_35}.  (Table 6 to 8 cannot be referred. Why??)
in the next section. 
To eliminate aliasing in the spherical harmonic expansion, the number of meridional and zonal grids are set to $1.5 L_{\rm max}$ and $3 L_{\rm max}$, respectively.
{\color{red}
Spatial resolutions are confirmed through kinetic and magnetic energy spectra.
Temporal resolutions and 
{\color{blue} 
computational
} 
duration are found to be sufficient through evaluation of temporal variations in kinetic and magnetic energies.
}
%}
