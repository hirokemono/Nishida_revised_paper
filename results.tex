\section{RESULTS}

\subsection{Estimation of the critical Rayleigh number}

To estimate the critical Rayleigh number ($Ra_{\rm crit}$), equations (\ref{eq:momentum})--(\ref{eq:heat}) without the Lorentz force term $\left(Pm E\right)^{-1} \left(\bvec{J} \times \bvec{B} \right)$, were solved as non-magnetic thermal convection simulations. The kinetic energy density was calculated for the average of $t / \tau_{\nu} = 4.5$ to 6 in viscous diffusion time as follows: 
%
\begin{equation}
E_{\rm kin} = \frac{1}{V_{S}} \int_{V_{S}} \frac{1}{2} \bvec{u}^{2} dV,
\label{eq:kinetic_energy}
\end{equation}
%
where $V_{S}$ is the volume of the spherical shell.

$E_{\rm kin}$ listed in Table~\ref{table:Rac} is calculated as a mean over a quasi-steady state, and is found to be linearly related to, as shown in Fig.~1.
The critical Rayleigh numbers are estimated as $Ra_{\rm crit} = 1.09 \tiems 10^5$, $0.72 \times 10^5$, and $0.56 \tiems 10^5$ in $r_i/r_o = 0.15$, $0.25$, and $0.35$, respectively, by the same method as Al-Shamali {\it et al.} \shortcite{Al-Shamali:2004}. 
The obtained values of $Ra_{\rm crit}$ were almost identical to those reported in Al-Shamali {\it et al.} \shortcite{Al-Shamali:2004} for the same parameters and conditions used in this study. 
$Ra_{\rm crit}$ is large when the aspect ratio is smaller, indicating that convection in a rotating, thick spherical shell requires a large buoyancy.
%
%
\begin{table}
\caption{Average kinetic energy density $E_{kin}$ at the quasi-steady state \\
for the thermal convection without the magnetic field}
\begin{center}
\begin{tabular}{|ccc||}
   \hline
  $r_{\rm i}/r_{\rm o}$ & $Ra[\times 10^5] $ &  $E_{\rm kin}$ \\
    \hline \hline
 0.15  &  1.0  &  $8.31 \times 10^{-6}$ \\
 0.15  &  1.2  &  5.72 \\
 0.15  &  1.25 &  8.15 \\
 0.15  &  1.3  &  10.62 \\
 0.15  &  1.35 &  13.16 \\
 0.15  &  1.4  &  15.80 \\
 0.15  &  1.45  &  18.54 \\
 \hline
%
 0.25  &  0.70 &  $8.55 \times 10^{-4}$ \\
 0.25  &  0.75 &  2.41 \\
 0.25  &  0.78  &  4.94 \\
 0.25  &  0.80  &  6.62 \\
 0.25  &  0.82  & 8.35 \\
 0.25  &  0.85  &  11.02 \\
 0.25  &  0.90  &  15.70 \\
 \hline
%
 0.35  &  0.55  &  $1.76 \times 10^{-4}$ \\
 0.35  &  0.58  &  2.38  \\
 0.35  &  0.60  &  4.54  \\
 0.35  &  0.62  &  6.76 \\
 0.35  &  0.65  &  10.22 \\
 0.35  &  0.67  &  12.60\\
 0.35  &  0.70  &  16.28 \\
 \hline
 \hline
\end{tabular}
\end{center}
\label{table:Rac}
\end{table}
%
\subsection{MHD simulation results}

We performed MHD dynamo simulations for various Rayleigh numbers and the radius ratios using Eqs (1)--(5). The magnetic energy density was calculated as follows:
%
\begin{equation}
E_{mag} = \frac{1}{V_{S}E Pm} \int_{V_{S}} \frac{1}{2} \bvec{B}^{2} dV.
\label{eq:magnetic_energy}
\end{equation}
%
Tables~\ref{table:Summary_15}, \ref{table:Summary_25}, and \ref{table:Summary_35} list results of MHD dynamo simulations. 
We performed respective numerical simulations for at least two magnetic diffusion times to assess whether the magnetic field was sustained or dissipated. 
For example, Fig.~2 shows the time evolution of the kinetic and magnetic energy density at $Ra/Ra_{\rm crit} = 2.8$ for $r_i/r_o = 0.25$. 
In this case, the magnetic field was sustained. 
We calculated the time average of the kinetic and magnetic energy density, as well as the dipolarity over a 0.5 magnetic diffusion time, at the end of each case (see shaded area in Fig.~2). The kinetic and magnetic energy density as a function of Rayleigh number is shown in Fig.~3.%, where the black, red, and blue points are the E_kin values in the non-MHD cases, E_kin values in the MHD cases, and E_mag values in the MHD cases, respectively. The “F” denotes the failed dynamo cases. 
% Reviewer # のコメントに従い削除
In each radius ratio case with a sustained magnetic field, the $E_{\rm kin}$ values in the MHD cases were smaller than the $E_{\rm kin}$ values in the corresponding non-MHD cases. 
These results show that the Lorentz force caused by the intense magnetic field disturbs convection.
There were differences among the three radius ratio cases. 
At $r_i/r_o = 0.15$, the $E_{\rm mag}$ values in the MHD cases were smaller than the $E_{\rm kin}$ values in the MHD cases for all cases. 
This trend is consistent with the results of Heimpel {\it et al.} \shortcite{Heimpel:2005}, whose simulations were performed around dynamo onset. 
At $r_i/r_o = 0.25$, the values of $E_{\rm mag}$ in the MHD cases were either smaller or larger than the $E_{\rm kin}$ values in the MHD. For cases of $Ra/Ra_{\rm crit}  = 3.6$ and $4.0$, the $E_{\rm mag}$ values were significantly smaller than those for the cases of other Rayleigh numbers. 
Although the trend in the magnetic energy spectrum did not change, there was a decrease in the amplitude. At $r_i/r_o = 0.35$, the values of $E_{\rm mag}$ in the MHD cases were larger than the values of $E_{\rm kin}$ in the MHD cases for almost all cases. 
From the above, it is not likely to sustain a strong magnetic field with a smaller inner core.
%
\begin{table*}
 \begin{center}
\caption{Time average of the kinetic energy $E_{\rm kin}$, magnetic energy $E_{\rm mag}$, dipolarity $f_{\rm dip}$, and Elsasser number $\Lambda$ for the cases with $r_{\rm i}/r_{\rm o} = 0.15$}
  \begin{tabular}{cccccccc}
      \hline
     $Ra[\times 10^3]$  &  $Ra/Ra_{\rm crit}$&  $E_{\rm kin}$  &  $E_{\rm mag}$ & $f_{\rm dip}$ & $f_{\rm mag\_fit}$ & $\Lambda_{\rm d}$\\
    \hline \hline
    $760$  & $7.0$ &  $823.7$ & $5.193$ & $-$ & $-$ & $-$\\
    $870$  & $8.0$ &  $801.4$ & $501.6$ & $0.494$ & $1.435$ & $0.105$\\
    $980$  & $9.0$ &  $846.37$ & $579.0$ & $0.516$ & $1.866$ &$0.116$\\
    $1100$  & $10.1$ &  $748.36$ & $444.7$ & $0.349$ & $0.860$ & $0.053$\\
    $1300$  & $11.9$ &  $1493$ & $300.5$ & $0.117$ & $0.322$ & $0.068$\\
    $1500$  & $13.8$ &  $1867$ & $135.6$ & $0.155$ & $0.391$ & $0.034$\\
    $1700$  & $15.6$ &  $2141$ & $234.3$ & $0.172$ & $0.420$ & $0.054$\\
     \hline
  \end{tabular}
 \end{center}
 \label{table:Summary_15}
 \end{table*}
 
\begin{table*}
\begin{center}
\caption{Time average of the kinetic energy $E_{\rm kin}$, magnetic energy $E_{\rm mag}$, dipolarity $f_{\rm dip}$, and Elsasser number $\Lambda$ for the cases with $r_{\rm i}/r_{\rm o} = 0.25$}
  \begin{tabular}{ccccccc}
    \hline
     $Ra[\times 10^3]$  &  $Ra/Ra_{\rm crit}$&  $E_{\rm kin}$  &  $E_{\rm mag}$ & $f_{\rm dip}$ & $f_{\rm mag\_fit}$ & $\Lambda_{\rm d}$\\
    \hline \hline    
    $140$  & $1.9$ &  $76.69$ & $1.491 \times 10^{-4}$ & $-$ & $-$ & $-$\\
    $160$  & $2.2$ &  $59.05$ & $958.6$ & $0.860$ & $2.116$ & $0.355$\\
    $180$  & $2.5$ &  $62.89$ & $1097$ & $0.867$ & $2.397$ & $0.410$\\
    $200$  & $2.8$ &  $85.49$ & $844.4$ & $0.784$ & $1.828$ & $0.323$\\
    $220$  & $3.1$ &  $103.9$ & $769.2$ & $0.757$ & $2.441$ & $0.283$\\
    $260$  & $3.6$ &  $236.4$ & $48.4$ & $0.644$ & $2.477$ & $0.021$\\
    $290$  & $4.0$ &  $289.3$ & $83.0$ & $0.620$ & $2.610$ & $0.035$\\
    $330$  & $4.6$ &  $248.6$ & $753.8$ & $0.602$ & $2.287$ & $0.277$\\
    $360$  & $5.0$ &  $297.7$ & $769.0$ & $0.562$ & $1.935$ & $0.224$\\
    $430$  & $6.0$ &  $455.6$ & $491.0$ & $0.522$ & $1.887$ & $0.174$\\
    $500$  & $6.9$ &  $642.2$ & $305.2$ & $0.456$ & $1.551$ & $0.130$\\
    $580$  & $8.1$ &  $906.8$ & $126.0$ & $0.412$ & $1.556$ & $0.059$\\
    $700$  & $9.7$ &  $1221$ & $25.4$ & $-$ & $-$ & $-$\\
    \hline
  \end{tabular}
 \end{center}
 \label{table:Summary_25}
\end{table*}
 
 \begin{table*}
% \begin{center}
\caption{Time average of the kinetic energy $E_{\rm kin}$, magnetic energy $E_{\rm mag}$, dipolarity $f_{\rm dip}$, and Elsasser number $\Lambda$ for the cases with $r_{\rm i}/r_{\rm o} = 0.35$}
  \begin{tabular}{ccccccc}
    \hline
     $Ra[\times 10^3]$  &  $Ra/Ra_{\rm crit}$&  $E_{\rm kin}$  &  $E_{\rm mag}$ & $f_{\rm dip}$ & $f_{\rm mag\_fit}$ & $\Lambda_{\rm d}$\\
    \hline \hline
      $84$    & $1.5$ &  $35.09$ & $2.376 \times 10^{-3}$ & $-$ & $-$ & $-$\\
     $110$  & $2.0$ &  $43.61$ & $819.6$ & $0.816$ & $4.713$ & $0.420$\\
     $140$  & $2.5$ &  $89.35$ & $1408$ & $0.724$ & $3.174$ & $0.519$\\
     $170$  & $3.0$ &  $106.8$ & $950.2$ & $0.739$ & $4.239$ & $0.407$\\
     $200$  & $3.6$ &  $136.3$ & $890.2$ & $0.692$ & $3.900$ & $0.399$\\
     $230$  & $4.1$ &  $193.1$ & $938.4$ & $0.632$ & $2.946$ & $0.421$\\
     $280$  & $5.0$ &  $311.9$ & $895.6$ & $0.556$ & $2.848$ & $0.383$\\
     $340$  & $6.1$ &  $508.7$ & $713.6$ & $0.480$ & $2.006$ & $0.294$\\
     $400$  & $7.1$ &  $837.6$ & $73.46$ & $0.376$ & $2.071$ & $0.035$\\
     $450$  & $8.0$ &  $1060$ & $10.61$ & $-$ & $-$ & $-$\\
    \hline
  \end{tabular}
% \end{center}
\label{table:Summary_35}
 \end{table*}
%
Fig.~4 shows the radial component of the magnetic field at the CMB and the equatorial cross-sections of the $z$-component of the vorticity and magnetic field are plotted at $Ra/Ra_{\rm crit} = 11.9$ for $r_i/r_o = 0.15$. 
The same plots at $Ra/Ra_{\rm crit} = 3.1$ for $r_i/r_o = 0.25$ and at $Ra/Ra_{\rm crit} = 3.0$ for $r_i/r_o = 0.35$ are shown in Figs~5 and 6, respectively. 
At the equatorial plane, the magnetic field is concentrated in the anti-cyclone columns to generate a dipolar field at $r_i/r_o = 0.25$ and $0.35$; intense magnetic patches are located near the tangent cylinder, which is an imaginary cylinder tangent to the inner-core equator and coaxial with the rotation axis. 
In the case of $r_i/r_o = 0.15$ case, strong convection is generated locally, where strong $B_Z$ convection is generated between the cyclonic and anti-cyclonic columns at the equatorial plane.
As these intense magnetic fields are not concentrated in the convection columns, the radial magnetic field at the CMB near the tangent cylinder has a quadrupolar (symmetric with respect to the equator) and is smaller than that of the cases with different aspect ratios.
%
%