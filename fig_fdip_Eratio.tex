\begin{figure}
\begin{center}
\[
\begin{array}{cc}
\mbox{$E = 10^{-3}$} &
\mbox{$E = 10^{-4}$} \\
\includegraphics*[width=75mm]{Figures/fdip_vs_Rm_Ek3.pdf} &
\includegraphics*[width=75mm]{Figures/fdip_vs_Rm_Ek4.pdf} \\
\includegraphics*[width=75mm]{Figures/ffit_vs_Rm_Ek3.pdf} &
\includegraphics*[width=75mm]{Figures/ffit_vs_Rm_Ek4.pdf} \\
\includegraphics*[width=75mm]{Figures/Eratio_vs_Rm_Ek3.pdf} &
\includegraphics*[width=75mm]{Figures/Eratio_vs_Rm_Ek4.pdf}
\end{array}
%\includegraphics*[width=80mm]{Figures/Fig9.png}
\]
\end{center}
\caption{The dipolarity $f_{\rm dip}$, the ratio of the dipolar magnetic energy density at the CMB to the extrapolation of the spherical harmonic degree $l = 1$, $f_{\rm mag\_fit}$, and the ratio of the magnetic to kinetic energy densities $E_{\rm mag} / E_{\rm kin}$ averaged over the spherical shells as a function of the magnetic Reynolds numnber $Rm$ for different geometries. $f_{\rm dip}$,  $f_{\rm mag\_fit}$, and $E_{\rm mag} / E_{\rm kin}$ are shown in the top, middle, and bottom panels, respectively. The settings for $E = 10^{-3}$ and $E = 10^{-4}$ are shown in the left and right panels, respectively. The blue, green and red symbols indicate the cases of $r_{i} / r_{o} = 0.15$, $0.25$ and $0.35$, respectively. 
}
\label{fig:fdip_fit_Eratio}
\end{figure}