\section{INTRODUCTION}
The Earth has an intrinsic magnetic field. The dynamo action of liquid iron alloy convection in the outer core generates the geomagnetic field. The fluid alloy gains buoyancy from the cooling of the Earth. Upon cooling, the inner core nucleated as liquid iron solidified from the center of the fluid core at high pressure. Compositional convection, which is associated with the growth of the inner core, is also a source of outer core convection. Recent thermochemical calculations suggest that the inner core which formed approximately one billion years ago and the inner core has been continually growing to its present size (Labrosse et al., 2001). Although the size of the inner core has been changing across the geological time scale, the intensity of the geomagnetic field described by the virtual dipole moment (VDM) has maintained its present intensity for more than 3.5 billion years based on paleomagnetic observations (Biggin et al., 2015); the geodynamo has been sustained during this period.

Previous studies have performed a number of numerical dynamo simulations under the assumption of the present geometry of the Earth’s core; the aspect ratio of the inner core radius, , to the outer core radius, , is = 0.35. For example, Christensen and Aubert (2006) revealed, in detail, sustained dynamo conditions for various control parameters. They clarified dipolar dynamo cases and non-dipolar dynamo cases. The dynamo regime changes from stable dipolar to reversing non-dipolar with increase of the Rayleigh numbers (Kutzner and Christensen, 2002; Olson et al., 2011).

In the smaller inner core setting, as compared with the present size, however, there have been a few attempts at a numerical dynamo. Some numerical dynamo studies have shown that the geometry effect on the dynamo regime is small. Hori et al. (2010) investigated the morphology of a magnetic field with fixed temperature (FT) and fixed heat flux (FF) boundary conditions for two spherical shell radius ratios: = 0.10 and 0.35. Regardless of the difference in radius ratios, they found that sustained dynamos were dipolar under the FF boundary condition and non-dipolar under the FT boundary condition. Driscoll (2016) carried out numerical simulations of geodynamo for eleven patterns of radius ratios in the range of 0.10 <  < 0.35 and core power derived from a thermal evolution model. Driscoll (2016) found that the total magnetic energy in a spherical shell increased with increase of  ratios and that sustained dynamos were characterized by a strong dipole magnetic field.

Other numerical dynamo studies have shown that the inner core size influences dipolar dominance. Heimpel et al. (2005) investigated dynamo onset conditions for six spherical shell radius ratios: 0.15 <  < 0.65. They found that the dipolar and total magnetic energy at the core–mantle boundary (CMB) decreases with decrease of  values for  < 0.45. Lhuillier et al. (2019) also reported on the effect of geometry on the dynamo regime. They performed chemically driven geodynamo simulations by changing ten patterns of radius ratios in the range of 0.10 <  < 0.44. They found that sustained magnetic fields were dipolar for  < 0.18 and  > 0.26, whereas they were less dipolar for 0.20 <  < 0.22. Although some studies have attempted to reveal the dependence of the dynamo regime on the spherical shell radius ratio, we do not yet fully understand how the morphology of the magnetic field is determined.

In recent numerical dynamos, dipolarity, , which is defined as the ratio of the dipole field strength to the total field strength at the CMB (Christensen and Aubert, 2006), has been widely used as an index for assessing the morphology of geomagnetic field. Christensen and Aubert (2006) mention that the magnetic field is dipolar-dominated when  exceeds 0.35. This criterion for the dynamo regime is valid when dynamos are categorized into large and small  groups (Soderlund et al., 2012). However, this criterion is not valid when dynamos are not categorized by dipolarity (Aubert et al., 2009). 

The observed dipolar-dominated geomagnetic field can be expressed in terms of the magnetic power spectrum at the Earth’s surface (Lowes, 1974) and at the CMB (Langel and Estes, 1982). While Kono anf Roberts (2002) compared a power  spectrum of the observed geomagnetic field with that of numerical dynamos, there was a lack of quantitative evaluation of the dipolar dominance. As the dipolarity has no information of magnetic power spectrum distribution in higher degrees, we require not only the dipolarity, but also another index that represents dipolar dominance assessed from the spectrum distribution.

Although numerical dynamo simulations are useful tools to investigate magnetic field intensity and structure in the past Earth environment, previous studies have not yet established the criterion to evaluate dipolar dominance. The purpose of this study is to investigate the dynamo conditions of a sustained dipolar or non-dipolar dynamo for some spherical shell radius ratios based on an evaluation of dipolar dominance. Hence, we carried out numerical simulations of geodynamo for three spherical shell radius ratios, i.e.,  = 0.15, 0.25, and 0.35. To focus on how convection occurs, the Rayleigh number () was only treated as a variable. The  is a parameter related to buoyancy, which is the driving force of convection. By simulating a wider range in the  than previous studies, we can compare cases of a small inner core size setting with those of the present size. A combination of the dipolarity at the CMB, as well as the magnetic energy spectrum at the CMB in the spherical harmonic degree expansion, reveal the range in the  in the sustained dipolar or non-dipolar dynamo for each radius ratio.
