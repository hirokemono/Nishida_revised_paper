\begin{figure}
\begin{center}
\[
\begin{array}{cc}
\mbox{$E = 10^{-3}$} &
\mbox{$E = 10^{-4}$} \\
\includegraphics*[width=70mm]{Figures/fdip_vs_RacRatio_Ek3.pdf} &
\includegraphics*[width=70mm]{Figures/fdip_vs_RacRatio_Ek4.pdf}
\end{array}
% \includegraphics*[width=80mm]{Figures/Fig7.png}
\]
\end{center}
% \caption{The dipolarity $f_{\rm dip}$ as a function of the magnetic Reynolds number $Rm$ for different spherical shell geometries. The cases with $E = 10^{-3}$ and $Pm = 5$ are plotted in the left panel, and the cases with $E = 10^{-4}$ and $Pm = 2$ are plotted in the right panel. The blue, green, and red marks indicate the cases of $r_{i} / r_{o} = 0.15$, 0,25, and 0.35, lfailed dynamo cases are plottes as negative value (-0.2). %The bar represents the standard deviation.
\caption{The dipolarity, $f_{\rm dip}$, as a function of $Ra / Ra_{\rm crit}$ for different spherical shell geometries.
The cases for $E = 10^{-3}$ and $Pm = 5$ are plotted in the left panel, and those for $E = 10^{-4}$ and $Pm = 2$ are plotted in the right panel.
The blue, green, and red symbols indicate the cases of $r_{i} / r_{o} = 0.15$, $0.25$ and $0.35$, respectively, and failed dynamo cases are plotted as negative value ($-0.2$).
}
\label{fig:fdip_vs_Racratio}
\end{figure}