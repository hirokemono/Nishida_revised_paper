% gjilguid2e.tex
% V2.0 released 1998 December 18
% V2.1 released 2003 October 7 -- Gregor Hutton, updated the web address for the style files.

\documentclass{gji}
\usepackage{timet,xcolor}

\title{Investigation of dipolar dominancy in geodynamo simulations with different inner core sizes}
\author{Y. Nishida$^1$, Y. Katoh$^1$, H. Matsui$^2$, M. Matsushima$^3$, T. Kera$^1$, and A. Kumamoto \\
  %\thanks{Pacific Region Office, GJI} \\
  $^1$ Department of Geophysics, Tohoku University, Sendai, Japan \\
  $^2$ Department of Earth and Planetary Sciences, University of California Davis, Davis, California, USA \\
  $^3$ Department of Earth and Planetary Sciences, Tokyo Institute of Technology, Tokyo, Japan
  }
\date{ }
\pagerange{\pageref{firstpage}--\pageref{lastpage}}
\volume{200}
\pubyear{1998}

%\def\LaTeX{L\kern-.36em\raise.3ex\hbox{{\small A}}\kern-.15em
%    T\kern-.1667em\lower.7ex\hbox{E}\kern-.125emX}
%\def\LATeX{L\kern-.36em\raise.3ex\hbox{{\Large A}}\kern-.15em
%    T\kern-.1667em\lower.7ex\hbox{E}\kern-.125emX}
% Authors with AMS fonts and mssymb.tex can comment out the following
% line to get the correct symbol for Geophysical Journal International.
\let\leqslant=\leq

\newtheorem{theorem}{Theorem}[section]

\begin{document}

\maketitle
%
\begin{summary}
The solid inner core of the Earth has been growing for approximately one billion years due to cooling of the Earth. The changing spherical shell geometry of the Earth’s core is likely to influence on the geodynamo driven by convective motions in the fluid outer core. To understand the geometry effect on the dynamo regime through evolution of the core, we perform numerical simulations of geodynamo with three spherical shell radius ratios: $r_{i}/r_{o} = 0.15$, 0.25, and 0.35, where $r_{i}$ and $r_{o}$ are the inner and outer core radii, respectively. To evaluate the morphology of the magnetic field, we examine two indices about dipole component dominance: (i) $f_{dip}$, dipolarity used to assess the relative strength of the dipole field at the core surface in numerical dynamo models; and (ii) $f_{mag\_fit}$, the ratio of magnetic energy density for the dipole component to that extrapolated from the magnetic power spectrum for the high degree components. We investigate the field morphology estimated from $f_{dip}$, and $f_{mag\_fit}$, and find that $f_{mag\_fit}$ is valid to determine the dynamo regime, even if $f_{dip}$ suggests a transition regime between dipolar and non-dipolar dominance. We also investigate the range of Rayleigh number for sustained dynamos based on $f_{dip}$ and $f_{mag\_fit}$, and find that the range of Rayleigh number for a dynamo characterized by a strong dipole field becomes narrower for a smaller inner core. The $f_{dip}$–dependence on the Rayleigh number for $r_{i}/r_{o} = 0.25$ and 0.35 is similar each other, whereas the $f_{mag\_fit}$–dependence for $r_{i}/r_{o} = 0.35$ is found to be relatively larger than that for $r_{i}/r_{o} = 0.25$. On the other hand, small values of  $f_{dip}$ and  $f_{mag\_fit}$ for $r_{i}/r_{o} = 0.15$ suggest that the dynamo regime is characterized not by a strong dipole field but by non-dipolar dominance. These results indicate that changes in the spherical shell radius ratio largely influence on the dynamo regime in numerical dynamos with a fixed temperature boundary condition.
\end{summary}
%
\section{INTRODUCTION}
The Earth has an intrinsic magnetic field. The dynamo action of liquid iron alloy convection in the outer core generates the geomagnetic field. The fluid alloy gains buoyancy from the cooling of the Earth. Upon cooling, the inner core nucleated as liquid iron solidified from the center of the fluid core at high pressure. Compositional convection, which is associated with the growth of the inner core, is also a source of outer core convection. Recent thermochemical calculations suggest that the inner core which formed approximately one billion years ago and the inner core has been continually growing to its present size \cite{Labrosse:2001}. Although the size of the inner core has been changing across the geological time scale, the intensity of the geomagnetic field described by the virtual dipole moment (VDM) has maintained its present intensity for more than 3.5 billion years based on paleomagnetic observations \cite{Biggin:2015}; the geodynamo has been sustained during this period.

Previous studies have performed a number of numerical dynamo simulations under the assumption of the present geometry of the Earth’s core; the aspect ratio of the inner core radius, $r_{i}$, to the outer core radius, $r_{o}$, is $r_{i} / r_{o} = 0.35$. For example, Christensen and Aubert \shortcite{Uli:2006}
revealed, in detail, sustained dynamo conditions for various control parameters. They clarified dipolar dynamo cases and non-dipolar dynamo cases. The dynamo regime changes from stable dipolar to reversing non-dipolar with increase of the Rayleigh numbers \cite{Kutzner:2002,Olson:2011}.

In the smaller inner core setting, as compared with the present size, however, there have been a few attempts at a numerical dynamo. Some numerical dynamo studies have shown that the geometry effect on the dynamo regime is small. Hori {\it et al.} \shortcite{Hori:2010} investigated the morphology of a magnetic field with fixed temperature (FT) and fixed heat flux (FF) boundary conditions for two spherical shell radius ratios:  $r_{i} / r_{o} = 0.10$ and 0.35. Regardless of the difference in radius ratios, they found that sustained dynamos were dipolar under the FF boundary condition and non-dipolar under the FT boundary condition. Driscoll \shortcite{Driscoll:2016} carried out numerical simulations of geodynamo for eleven patterns of radius ratios in the range of $0.10 < r_{i} / r_{o} < 0.35$ and core power derived from a thermal evolution model. Driscoll \shortcite{Driscoll:2016} found that the total magnetic energy in a spherical shell increased with increase of the ratios $r_{i} / r_{o}$ and that sustained dynamos were characterized by a strong dipole magnetic field.

Other numerical dynamo studies have shown that the inner core size influences dipolar dominance. Heimpel {\it et al.} \shortcite{Heimpel:2005} investigated dynamo onset conditions for six spherical shell radius ratios: $0.15 < r_{i} / r_{o} < 0.65$. They found that the dipolar and total magnetic energy at the core–mantle boundary (CMB) decreases with decrease of $r_{i} / r_{o}$ values for  $r_{i} / r_{o} < 0.45$. Lhuillier {\it et al.} \shortcite{Lhuillier:2019} also reported on the effect of geometry on the dynamo regime. They performed chemically driven geodynamo simulations by changing ten patterns of radius ratios in the range of $0.10 < r_{i} / r_{o} < 0.44$. They found that sustained magnetic fields were dipolar for  $r_{i} / r_{o} < 0.18$ and  $r_{i} / r_{o} > 0.26$, whereas they were less dipolar for $0.20 < r_{i} / r_{o} < 0.22$. Although some studies have attempted to reveal the dependence of the dynamo regime on the spherical shell radius ratio, we do not yet fully understand how the morphology of the magnetic field is determined.

In recent numerical dynamos, dipolarity, $f_{dip}$, which is defined as the ratio of the dipole field strength to the total field strength at the CMB \cite{Uli:2006}, has been widely used as an index for assessing the morphology of geomagnetic field. Christensen and Aubert \shortcite{Uli:2006} mention that the magnetic field is dipolar-dominated when $f_{dip}$ exceeds 0.35. This criterion for the dynamo regime is valid when dynamos are categorized into large and small $f_{dip}$ groups \cite{Soderlund:2012}. However, this criterion is not valid when dynamos are not categorized by dipolarity \cite{Aubert:2009}. 

The observed dipolar-dominated geomagnetic field can be expressed in terms of the magnetic power spectrum at the Earth’s surface \cite{Lowes:1974} and at the CMB \cite{Langel:1982}. While Kono anf Roberts \shortcite{Kono:2002} compared a power  spectrum of the observed geomagnetic field with that of numerical dynamos, there was a lack of quantitative evaluation of the dipolar dominance. As the dipolarity has no information of magnetic power spectrum distribution in higher degrees, we require not only the dipolarity, but also another index that represents dipolar dominance assessed from the spectrum distribution.

Although numerical dynamo simulations are useful tools to investigate magnetic field intensity and structure in the past Earth environment, previous studies have not yet established the criterion to evaluate dipolar dominance. The purpose of this study is to investigate the dynamo conditions of a sustained dipolar or non-dipolar dynamo for some spherical shell radius ratios based on an evaluation of dipolar dominance. Hence, we carried out numerical simulations of geodynamo for three spherical shell radius ratios, i.e., $r_{i} / r_{o} = 0.15$, 0.25, and 0.35. To focus on how convection occurs, the Rayleigh number ($Ra$) was only treated as a variable. The $Ra$ is a parameter related to buoyancy, which is the driving force of convection. By simulating a wider range in the $Ra$ than previous studies, we can compare cases of a small inner core size setting with those of the present size. A combination of the dipolarity at the CMB, as well as the magnetic energy spectrum at the CMB in the spherical harmonic degree expansion, reveal the range in the $Ra$ in the sustained dipolar or non-dipolar dynamo for each radius ratio.

%
\begin{thebibliography}{}
%
{\color{red}
\bibitem[\protect\citename{Alken {\it et al.} }2021] {Alken:2021}
Alken, P., Thebault, E., Beggan, C. D., Amit, H., Aubert, J., Baerenzung, J., Bondar, T. N., Brown, W. J., Califf, S., Chambodut, A., Chulliat, A., Cox, G. A., Finlay, C. C., Fournier, A., N., Gillet, Grayver, A., Hammer, M. D., Holschneider, M., Huder, L., Hulot, G., Jager, T., Kloss, C., Korte, M., Kuang,W., Kuvshinov, A., Langlais, B., Leger, J.-M., Lesur, V., Livermore, P. W., Lowes, F. J., Macmillan, S., Magnes, W., Mandea, M., Marsal, S., Matzka, J., Metman, M. C., Minami, T., Morschhauser, A., Mound, J. E., Nair, M., Nakano, S., Olsen, N., Pavon-Carrasco, F. J., Petrov, V. G., Ropp, G., Rother, M., Sabaka, T. J., Sanchez, S., Saturnino, D., Schnepf, N., R., Shen, X., Stolle, C., Tangborn, A., Toffner-Clausen, L., Toh, H., Torta, J. M., Varner, J., Vervelidou, F., Vigneron, P., Wardinski, I., Wicht, J., Woods, A., Yang, Y., Zeren, Z. and Zhou, B., 2021. International Geomagnetic Reference Field: the thirteenth generation, {\it Earth Planets Space}, {\bf 73}:49, doi: 10.1186/s40623-020-01288-x.
}
%
\bibitem[\protect\citename{Al-Shamali {\it et al.} }2004] {Al-Shamali:2004}
Al-Shamali, F. M., Heimpel, M. H., \& Aurnou, J. M., 2004. Varying the spherical shell geometry in rotating thermal convection, {\it Geophys.\  Astrophys.\ Fluid Dyn.}, {\bf 98}(2), 153--169, doi:10.1080/0309192041000165928.
%
\bibitem[\protect\citename{Aubert {\it et al.} }2009] {Aubert:2009}
Aubert, J., Labrosse, S., \& Poitou, C., 2009, Modelling the paleo-evolution of the geodynamo, {\it Geophys.\ J. Int.}, {\bf 179}, 1414--1428, doi: 10.1111/j.1365-246X.2009.04361.x.
%
\bibitem[\protect\citename{Biggin {\it et al.} }2015] {Biggin:2015}
Biggin, A. J., Piispa, E. J., Pesonen, L. J., Holme, R., Paterson, G. A., Veikkolainen, T., \& Tauxe, L., 2015. Palaeomagnetic field intensity variations suggest Mesoproterozoic inner-core nucleation, {\it Nature}, {\bf 526}, 245–248.
%
\bibitem[\protect\citename{Christensen {\it et al.} }2001] {Uli:2001}
Christensen, U. R., Aubert J., Cardin, P., Dormy, E., Gibbon, S., Glatzmaier, G. A., Grote, E., Honkura, Y., Jones, C., Kono, M., Matsushima, M., Sakuraba, A., Takahashi, F., Tilgner, A., Wicht, J. \& Zhnag, K., 2001. A numerical dynamo benchmark, \pepi {\bf 128}, 25--34, doi:10.1016/S0031-9201(01)00275-8.
%
\bibitem[\protect\citename{Christensen \& Aubert, }2006] {Uli:2006}
Christensen, U. R. \& Aubert, J., 2006. Scaling properties of convection-driven dynamos in rotating spherical shells and application to planetary magnetic fields, \gji {\bf 166}(1), 97--144, doi:10.1111/j.1365-246X.2006.03009.x
%
\bibitem[\protect\citename{Driscoll }2016] {Driscoll:2016}
Driscoll, P. E., 2016. Simulating 2 Ga of geodynamo history, \grl {\bf 43}, 5680--5687, doi:10.1002/2016GL068858.
%
{\color{red}
\bibitem[\protect\citename{Gastine {\it et al.}, }2016] {Gastine:2016}
Gastine, T., Wicht, J., \& Aubert, J., 2016. Scaling regimes in spherical shell rotating convection, {\it J. Fluid Mech.}, {\it 808}, 690--732, doi:10.1017/jfm.2016.659.
}
%
\bibitem[\protect\citename{Heimpel {\it et al.} }2005] {Heimpel:2005}
Heimpel, M. H., Aurnou, J. M., Al-Shamali, F. M. \& Perez N. G., 2005. A numerical study of dynamo action as a function of spherical shell geometry, {\it Earth planet.\ Sci.\ Lett.}, {\bf 236}, 542--557, doi:10.1016/j.epsl.2005.04.032.
%
\bibitem[\protect\citename{Hori {\it et al.} }2010] {Hori:2010}
Hori, K., Wicht, J. \& Christensen U. R., 2010. The effect of thermal boundary conditions on dynamo driven by internal heating, {\it Phys.\ Earth planet.\ Inter.}, {\bf 182}, 85--97, doi:10.1016/j.pepi.2010.06.011.
%
\bibitem[\protect\citename{Kono \& Roberts }2002] {Kono:2002}
Kono, M. \& Roberts, P. H., 2002. Recent geodynamo simulations and observations of the geomagnetic field, {\it Rev.\ Geophys.}, {\bf 40}(4), 1--53, doi:10.1029/2000RG000102.
%
\bibitem[\protect\citename{Kutzner and Christensen, }2002] {Kutzner:2002}
Kutzner, C. \& Christensen, U. R., 2002. From stable dipolar towards reversing numerical dynamos, {\it Phys.\ Earth planet.\ Inter.}, {\bf 131}(1), 29--45, doi:10.1016/S0031-9201(02)00016-X.
%
\bibitem[\protect\citename{Labrosse {\it et al.} }2001] {Labrosse:2001}
Labrosse, S., Poirier, J.-P., \& Le Mou{\"e}l, J.-L., 2001. The age of the inner core, {\it Earth Planet.\ Sci.\ Lett.}, {\bf 190}, 111--123.
%
\bibitem[\protect\citename{Langel \& Estes }1982] {Langel:1982}
Langel, R. A. \& Estes, R. H., 1982. A geomagnetic field spectrum, {\it Geophys.\ Res.\ Lett.}, {\bf 9}(4), 250--253.
%
\bibitem[\protect\citename{Lhuillier {\it et al.} }2019] {Lhuillier:2019}
Lhuillier, F., Hulot, G., Gallet, Y. \& Schwaiger, T., 2019. Impact of inner-core size on the dipole field behaviour of numerical dynamo simulations, \gji {\bf 218}, 179--189, doi:10.1093/gji/ggz146.
%
\bibitem[\protect\citename{Long {\it et al.} }2020] {Long:2020}
Long, R. S., Mound, J. E., Davis, C. J., \& Tobias, S. M., 2020. Scaling behavior in spherical shell rotating convection with fixed-flux thermal boundary conditions, {\it J. Fluid Mech.}, {\bf 889}, doi:10.1017/jfm.2020.67.
%
\bibitem[\protect\citename{Lowes }1974] {Lowes:1974}
Lowes, F. J., 1974. Spatial power spectrum of the main geomagnetic field, and extrapolation to the core, \gji {\bf 36(3)}, 717--730, doi:10.1111/j.1365-246X.1974.tb00622.x.
%
\bibitem[\protect\citename{Matsui {\it et al.} }2014] {Matsui:2014}
Matsui, H., King, E., \& Buffett, B., 2014. Multiscale convection in a geodynamo simulation with uniform heat flux along the outer boundary, {\it Geochem.\ Geophys.\ Geosys.}, {\bf 15},  3212--3225, doi:10.1002/2014GC005432.
%
\bibitem[\protect\citename{Merrill {\it et al.} }1996] {Merrill:1996}
Merrill, R. T., McElhinny, M. W., \& McFadden, P. L., 1996. The Magnetic Field of the Earth: Paleomagnetism, the Core, and the Deep Mantle, Academic Press.
%
\bibitem[\protect\citename{Olson {\it et al.} }2011] {Olson:2011}
Olson, L. P., Glatzmaier, G. A. \& Coe, R. S., 2011. Complex polarity reversals in a geodynamo moidel, {\it Earth planet.\ Sci.\ Lett.}, {\bf 304}(1)--(2), 168--179, doi:10.1016/j.epsl.2011.01.031.
%
\bibitem[\protect\citename{Soderlund {\it et al.} }2012] {Soderlund:2012}
Soderlund, K. M., King, E. M. \& Aurnou, J. M., 2012. The influence of magnetic field in planetary dynamo models, {\it Earth planet.\ Sci.\ Lett.}, {\bf 333--334}, 9--20, doi:10.1016/j.epsl.2012.03.038.
%
\bibitem[\protect\citename{Thebault {\it et al.} }2015] {Thebault:2015}
Thebault, E., Finlay, C. C., Beggan, C. D., Alken P., Aubert, J., Barrois, O., Bertrand, F., Bondar, T., Boness, A., Brocco, L., Canet, E., Chambodut, A., Chulliat, A., Coisson, P., Civet, F., Du, A., Fournier, A., Fratter, I., Gillet, N., Hamilton, B., Hamoudi, M., Hulot, G., Jager, T., Korte, M., Kuang, W., Lalanne, X., Langlais, B., Leger, J.-M., Lesur V., Lowes, F. J. Macmillan, S., Mandea, M., Manoj, C., Maus, S., Olsen, N., Petrov, V., Ridley, V., Rother, M., Sabaka, T. J., Saturnino, D., Schachtschneider, R., Sirol, O., Tangborn, A., Thomson, A., Toffner-Clausen, L., Vigneron, P., Wardinski, I., \& Zvereva, T., 2015. International Geomagnetic Reference Field: the 12th generation, {\it Earth Planets Space}, {\bf 67}:79, doi:10.1186/s40623-015-0228-9.
%

\end{thebibliography}

%
\end{document}
