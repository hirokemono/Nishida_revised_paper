% V2.0 released 1998 December 18
% V2.1 released 2003 October 7 -- Gregor Hutton, updated the web address for the style files.

% \documentclass{gji}
\documentclass[mreferee]{gji}
\usepackage{timet,xcolor,graphicx}

\title[Dipolar dominancy in geodynamo with different inner core sizes]
{Investigation of dipolar dominancy in geodynamo simulations with different inner core sizes}
\author[Y. Nishida et al.]
{Yuki Nishida$^1$, Yuto Katoh$^1$, Hiroaki Matsui$^2$, Masaki Matsushima$^3$,\\ 
{\rm \LARGE
Takumi Kera$^1$, and Atsushi Kumamoto$^1$ }\\
  %\thanks{Pacific Region Office, GJI} \\
  $^1$ Department of Geophysics, Tohoku University, Sendai, Japan \\
  $^2$ Department of Earth and Planetary Sciences, University of California Davis, Davis, California, USA \\
  $^3$ Department of Earth and Planetary Sciences, Tokyo Institute of Technology, Tokyo, Japan
  }
\date{ }
\pagerange{\pageref{firstpage}--\pageref{lastpage}}
\volume{23X}
\pubyear{2022}

%\def\LaTeX{L\kern-.36em\raise.3ex\hbox{{\small A}}\kern-.15em
%    T\kern-.1667em\lower.7ex\hbox{E}\kern-.125emX}
%\def\LATeX{L\kern-.36em\raise.3ex\hbox{{\Large A}}\kern-.15em
%    T\kern-.1667em\lower.7ex\hbox{E}\kern-.125emX}
% Authors with AMS fonts and mssymb.tex can comment out the following
% line to get the correct symbol for Geophysical Journal International.
\let\leqslant=\leq

\newtheorem{theorem}{Theorem}[section]

\input{mathdef.tex}
\begin{document}

\maketitle
%
\begin{summary}
The solid inner core of the Earth has been growing for approximately one billion years due to cooling of the Earth.
The changing spherical shell geometry of the Earth’s core is likely to influence the geodynamo driven by convective motions in the fluid outer core.
To understand the geometry effect on the dynamo regime through evolution of the core, we perform numerical simulations of geodynamo with three spherical shell radius ratios: $r_{i}/r_{o} = 0.15$, $0.25$, and $0.35$, where $r_{i}$ and $r_{o}$ are the inner and outer core radii, respectively.
% To evaluate the morphology of the magnetic field, we examine two indices for dipole component dominance: (i) $f_{\rm dip}$, dipolarity used to assess the relative strength of the dipole field at the core surface in numerical dynamo models; and (ii) $f_{\rm mag\_fit}$, the ratio of magnetic energy density for the dipole component to that extrapolated from the magnetic power spectrum for the high degree components.
To evaluate the morphology of the magnetic field, we examine two indices for dipole component dominance.
{\color{red}One is $f_{\rm dip}$, the dipolarity used to assess the relative strength of the dipole field at the core surface in numerical dynamo models, and the other is $f_{\rm mag\_fit}$, the ratio of magnetic energy for the dipole component to that obtained from extrapolation of the magnetic energy spectrum for the non-dipole components.}
% We investigate the field morphology estimated from $f_{\rm dip}$, and $f_{\rm mag\_fit}$, and find that $f_{\rm mag\_fit}$ is valid to determine the dynamo regime, even if $f_{\rm dip}$ suggests a transition regime between dipolar and non-dipolar dominance.
{\color{red}We find that $f_{\rm mag\_fit}$ is valid to determine the dynamo regime, even if $f_{\rm dip}$ suggests a transition regime between dipolar and non-dipolar dominance.}
We also investigate the range of Rayleigh number for sustained dynamos based on $f_{\rm dip}$ and $f_{\rm mag\_fit}$, and find that the range of Rayleigh number for a dynamo characterized by a strong dipole field becomes narrower for a smaller inner core.
The $f_{\rm dip}$–dependence on the Rayleigh number for $r_{i}/r_{o} = 0.25$ and 0.35 is similar to each other, whereas the $f_{\rm mag\_fit}$–dependence for $r_{i}/r_{o} = 0.35$ is found to be relatively larger than that for $r_{i}/r_{o} = 0.25$. On the other hand, small values of  $f_{\rm dip}$ and  $f_{\rm mag\_fit}$ for $r_{i}/r_{o} = 0.15$ suggest that the dynamo regime is characterized not by a strong dipole field but by non-dipolar dominance.
These results indicate that changes in the spherical shell radius ratio largely influence on the dynamo regime in numerical dynamos with a fixed temperature boundary condition.
\end{summary}
%
\begin{keywords}
Dynamo; Theories and simulations; Magnetic field variations through time; Numerical solutions
\end{keywords}
%
\section{INTRODUCTION}
The Earth has an intrinsic magnetic field. The dynamo action of liquid iron alloy convection in the outer core generates the geomagnetic field. The fluid alloy gains buoyancy from the cooling of the Earth. Upon cooling, the inner core nucleated as liquid iron solidified from the center of the fluid core at high pressure. Compositional convection, which is associated with the growth of the inner core, is also a source of outer core convection. Recent thermochemical calculations suggest that the inner core which formed approximately one billion years ago and the inner core has been continually growing to its present size (Labrosse et al., 2001). Although the size of the inner core has been changing across the geological time scale, the intensity of the geomagnetic field described by the virtual dipole moment (VDM) has maintained its present intensity for more than 3.5 billion years based on paleomagnetic observations (Biggin et al., 2015); the geodynamo has been sustained during this period.

Previous studies have performed a number of numerical dynamo simulations under the assumption of the present geometry of the Earth’s core; the aspect ratio of the inner core radius, , to the outer core radius, , is = 0.35. For example, Christensen and Aubert (2006) revealed, in detail, sustained dynamo conditions for various control parameters. They clarified dipolar dynamo cases and non-dipolar dynamo cases. The dynamo regime changes from stable dipolar to reversing non-dipolar with increase of the Rayleigh numbers (Kutzner and Christensen, 2002; Olson et al., 2011).

In the smaller inner core setting, as compared with the present size, however, there have been a few attempts at a numerical dynamo. Some numerical dynamo studies have shown that the geometry effect on the dynamo regime is small. Hori et al. (2010) investigated the morphology of a magnetic field with fixed temperature (FT) and fixed heat flux (FF) boundary conditions for two spherical shell radius ratios: = 0.10 and 0.35. Regardless of the difference in radius ratios, they found that sustained dynamos were dipolar under the FF boundary condition and non-dipolar under the FT boundary condition. Driscoll (2016) carried out numerical simulations of geodynamo for eleven patterns of radius ratios in the range of 0.10 <  < 0.35 and core power derived from a thermal evolution model. Driscoll (2016) found that the total magnetic energy in a spherical shell increased with increase of  ratios and that sustained dynamos were characterized by a strong dipole magnetic field.

Other numerical dynamo studies have shown that the inner core size influences dipolar dominance. Heimpel et al. (2005) investigated dynamo onset conditions for six spherical shell radius ratios: 0.15 <  < 0.65. They found that the dipolar and total magnetic energy at the core–mantle boundary (CMB) decreases with decrease of  values for  < 0.45. Lhuillier et al. (2019) also reported on the effect of geometry on the dynamo regime. They performed chemically driven geodynamo simulations by changing ten patterns of radius ratios in the range of 0.10 <  < 0.44. They found that sustained magnetic fields were dipolar for  < 0.18 and  > 0.26, whereas they were less dipolar for 0.20 <  < 0.22. Although some studies have attempted to reveal the dependence of the dynamo regime on the spherical shell radius ratio, we do not yet fully understand how the morphology of the magnetic field is determined.

In recent numerical dynamos, dipolarity, , which is defined as the ratio of the dipole field strength to the total field strength at the CMB (Christensen and Aubert, 2006), has been widely used as an index for assessing the morphology of geomagnetic field. Christensen and Aubert (2006) mention that the magnetic field is dipolar-dominated when  exceeds 0.35. This criterion for the dynamo regime is valid when dynamos are categorized into large and small  groups (Soderlund et al., 2012). However, this criterion is not valid when dynamos are not categorized by dipolarity (Aubert et al., 2009). 

The observed dipolar-dominated geomagnetic field can be expressed in terms of the magnetic power spectrum at the Earth’s surface (Lowes, 1974) and at the CMB (Langel and Estes, 1982). While Kono anf Roberts (2002) compared a power  spectrum of the observed geomagnetic field with that of numerical dynamos, there was a lack of quantitative evaluation of the dipolar dominance. As the dipolarity has no information of magnetic power spectrum distribution in higher degrees, we require not only the dipolarity, but also another index that represents dipolar dominance assessed from the spectrum distribution.

Although numerical dynamo simulations are useful tools to investigate magnetic field intensity and structure in the past Earth environment, previous studies have not yet established the criterion to evaluate dipolar dominance. The purpose of this study is to investigate the dynamo conditions of a sustained dipolar or non-dipolar dynamo for some spherical shell radius ratios based on an evaluation of dipolar dominance. Hence, we carried out numerical simulations of geodynamo for three spherical shell radius ratios, i.e.,  = 0.15, 0.25, and 0.35. To focus on how convection occurs, the Rayleigh number () was only treated as a variable. The  is a parameter related to buoyancy, which is the driving force of convection. By simulating a wider range in the  than previous studies, we can compare cases of a small inner core size setting with those of the present size. A combination of the dipolarity at the CMB, as well as the magnetic energy spectrum at the CMB in the spherical harmonic degree expansion, reveal the range in the  in the sustained dipolar or non-dipolar dynamo for each radius ratio.

%
\section{METHOD}

A numerical geodynamo model is given by an electrically conducting Boussinesq fluid in a rotating spherical shell. The governing equations of the geodynamo in the outer core are described by the momentum equation, continuity equation, heat equation, magnetic induction equation, and Gauss's law for the magnetic field, which are respectively given as
%
\begin{equation}
\begin{array}{l}
\displaystyle
\frac{\partial \bvec{u}}{\partial t} + \left(\bvec{\omega} \times \bvec{u}\right)
 = - \nabla \left(P+\frac{1}{2}u^{2} \right) - \nabla \times \nabla \times \bvec{u}
\\
\displaystyle
\hspace*{3em}
      - \frac{2}{E} \left(\hat{\bvec{z}} \times \bvec{u} \right)
  + \frac{Ra}{Pr} T \frac{\bvec{r}}{r_{o}}
        + \frac{1}{Pm E} \left(\bvec{J} \times \bvec{B} \right),
\label{eq:momentum}
\end{array}
\end{equation}
%
\begin{equation}
\nabla \cdot \bvec{u} = 0, 
\label{eq:conservation}
\end{equation}
%
\begin{equation}
\frac{\partial T}{\partial t} + \left(\bvec{u} \cdot \nabla \right) T
 = \frac{1}{Pr} \nabla^{2} T,
\label{eq:heat}
\end{equation}
%
\begin{equation}
 \frac{\partial \bvec{B}}{\partial t}
 = -\frac{1}{Pm}  \nabla \times \nabla \times \bvec{B}
       + \nabla \times \left(\bvec{u} \times \bvec{B} \right),
\label{eq:induction}
\end{equation}
%
and
\begin{equation}
\nabla \cdot \bvec{B} = 0,
\label{eq:Gauss_B}
\end{equation}
%
where $\bvec{u}$, $t$, $P$, $T$, $\bvec{B}$, $\bvec{r}$, and $\hat{\bvec{z}}$ are the velocity, time, reduced pressure, temperature, magnetic field, position vector, and unit vector along the rotation axis, respectively.
% $\bvec{\omega} = \nabla \times \bvec{u}$ and $\bvec{J} = \nabla \times \bvec{B} / \mu_0$ are the vorticity and current density, respectively.
In eq.~(\ref{eq:momentum}), $\bvec{\omega} = \nabla \times \bvec{u}$ is the vorticity, and $\bvec{J} = \nabla \times \bvec{B} / \mu_0$ in the dimensional form is the current density, where $\mu_0$ is the magnetic permeability of vacuum.
In eqs~(\ref{eq:momentum})--(\ref{eq:Gauss_B}), the length and temperature are normalised by the outer core thickness, $L = r_{o} - r_{i}$ and average temperature difference between the inner core boundary (ICB) and the CMB, $\Delta T$, respectively. 
The time is normalised by the kinematic viscosity diffusion time, $\tau_{\nu}  = L^{2} / \nu$, where $\nu$ is the kinetic viscosity, and the magnetic field is normalised by $\sqrt{\rho_{0} \mu_{0} \eta \Omega}$, where $\rho_{0}$, $\eta$, and $\Omega$ are the average density of core fluid, the magnetic diffusivity, and the angular velocity of system's rotation, respectively.
The Rayleigh number, $Ra$, the Ekman number, $E$, the Prandtl number, $Pr$, and the magnetic Prandtl number, $Pm$, are respectively defined as follows:
%
\begin{equation}
Ra = \displaystyle{ \frac{\alpha g_o \Delta T L^{3}}{ \kappa \nu} }, 
E  = \displaystyle{ \frac{\nu}{\Omega L^{2}} },
Pr = \displaystyle{ \frac{\nu}{\kappa} }, 
\mbox{ and }
Pm = \displaystyle{ \frac{\nu}{\eta} },
\label{eq:dimensionless}
\end{equation}
%
where, $\alpha$, $g_o$, and $\kappa$ are the thermal expansion coefficient, the gravitational acceleration at the CMB, and the thermal diffusivity, respectively.

% We used the numerical dynamo code Calypso \cite{Matsui:2014}. 
% In Calypso, the spherical harmonic expansion method is used in the horizontal discretization, and the second-order finite differences are used in the radial discretization. 
% For the time integrations, the Crank-Nicolson method was used in the linear diffusive terms and the second order Adams-Bashforth method was used in the other terms.
%
% MM back to black
We use a numerical dynamo code Calypso \cite{Matsui:2014}, in which numerical simulations are carried out in the spherical coordinates, $(r, \theta, \phi)$.
Solenoidal vector fields, $\bvec{u}$ and $\bvec{B}$, are decomposed into the toroidal and poloidal constituents {\color{red}as
%
\begin{equation}
\bvec{u}(\bvec{r}, t) = \nabla \times (u_T (\bvec{r}, t) \hat{\bvec{r}} ) + \nabla \times \nabla \times (u_S (\bvec{r}, t) \hat{\bvec{r}} ) ,
\label{eq:u_uT_uS}
\end{equation}
%
\begin{equation}
\bvec{B}(\bvec{r}, t) = \nabla \times (B_T (\bvec{r}, t) \hat{\bvec{r}} ) + \nabla \times \nabla \times (B_S (\bvec{r}, t) \hat{\bvec{r}} ) ,
\label{eq:B_BT_BS}
\end{equation}
%
where $\hat{\bvec{r}}$ is the radial unit vector.
Todoidal and poloidal scalar functions for the velocity and magnetic fields, $u_T(\bvec{r}, t)$, $u_S(\bvec{r}, t)$, $B_T(\bvec{r}, t)$, and $B_S(\bvec{r}, t)$ are expanded into spherical harmonics.
As an example, $u_T (\bvec{r}, t)$ is expanded as
%
\begin{equation}
u_T (\bvec{r}, t) = \sum_{l = 1}^{L_{\rm max}} \sum_{m=-l}^{l} u_{T l}^{\ m} (r, t) Y_l^{|m|} (\theta, \phi) ,
\label{eq:uT_Ylm}
\end{equation}
%
where $L_{\rm max}$ is the truncation of spherical harmonics, and
%
\begin{equation}
Y_l^{|m|} (\theta, \phi) = \left\{
 \begin{array}{ll}
 P_l^m(\cos\theta)\cos m\phi & (m = 0, 1, 2, \cdots, l)
 \\
 P_l^{|m|}(\cos\theta)\sin |m|\phi & (m = -1, -2, \cdots, -l) .
 \end{array}
\right.
\label{eq:def_of_Ylm}
\end{equation}
%
$P_l^m(\cos\theta)$ is a Schmidt-normalised associated Legendre polynomial with degree $l$ and order $m$.
The temperature, $T(\bvec{r}, t)$, is also expanded into spherical harmonics as
%
\begin{equation}
T (\bvec{r}, t) = \sum_{l = 0}^{L_{\rm max}} \sum_{m=-l}^{l} T_l^m(r, t) Y_l^{|m|} (\theta, \phi) .
\label{eq:T_Ylm}
\end{equation}
%
}

The radial grid points are defined as
%
\begin{equation}
r_n = r_i + \frac{r_o - r_i}{2} \left\{ 1 - \cos \left( \pi \frac{n-1}{N_r-1} \right) \right\} ~~\;\;\;\; (n = 1, \cdots , N_r) ,
\label{eq:def_of_rn}
\end{equation}
%
where $N_r$ is the number of radial grids, and spatial resolution near the boundary surfaces is expected to be high enough.
The second-order finite differences are used for the partial differentiation in the radial direction.
For the time integration, the Crank-Nicolson method is used in the diffusion and other linear terms, and the second order Adams-Bashforth method is used in the other terms.

As the initial condition, temperature perturbation is applied to all the sectorial modes $(l = m)$. 
The initial magnetic field is set as an axial dipole field and zonal toroidal field with degree $l=2$ adopted by Christensen {\it et al.} \shortcite{Uli:2001}. 
For the boundary condition, the non-dimensional temperatures at the CMB and ICB are fixed as $T(r_{o}) = 0$ and  $T(r_{i}) = 1$, respectively. 
The mantle and inner core are assumed to be co-rotating, and a non-slip boundary ($\bvec{u} = \bvec{0}$) is applied to the CMB and ICB. 
The mantle and inner core are also assumed to be electrically insulated, and the magnetic field at the boundaries is connected to the potential field.
% In the parameter setting, 
{\color{blue}
We choose two parameter settings; % for the Ekman, Parndtl, and magnetic Prandtl numbers. 
in one setting, the Ekman, Prandtl, and magnetic Prandtl numbers are fixed at $E = 10^{-3}$, $Pr = 1$, and $Pm = 5$, and in the other, $E = 10^{-4}$, $Pr = 1$, and $Pm = 2$ are chosen. 
To investigate the effects of different inner core sizes, the ratios of the inner core radius to the outer core radius are set as $r_{i} / r_{o} = 0.15$, $0.25$ and $0.35$ for each parameter setting, and $Ra$ is also changed in each setting. 
Parameters and spatial resolution, $L_{\rm max}$ and $N_{r}$, are listed in Tables \ref{table:Summary_15} to 8 
% \ref{table:SummaryEk4_35}.  (Table 6 to 8 cannot be referred. Why??)
in the next section. 
To eliminate aliasing in the spherical harmonic expansion, the number of meridional and zonal grids are set to $1.5 L_{\rm max}$ and $3 L_{\rm max}$, respectively.
}
%$Ra$ is changed among the cases; the Ekman, Prandtl, and magnetic Prandtl numbers are fixed at $E = 10^{-3}$, $Pr = 1$, and $Pm = 5$ in all simulation cases. 
%The truncation of the spherical harmonics and the radial grid points were set to $L_{\rm max} = 47$ and $N_{r} = 63$, respectively. 
%To eliminate aliasing in the spherical harmonic expansion, horizontal grids are set to $(N_{\theta}, N_{\phi}) = (72, 144)$.
% To investigate the effects of different inner core sizes, the spherical shell radius ratios of the inner core radius to the outer core radius are set as $r_{i} / r_{i} = 0.15$, 0.25, and 0.35. 
% First, we perform numerical simulations of non-magnetic thermal convection in rotating spherical shells to estimate the critical Rayleigh number, $Ra_{\rm crit}$, for the onset of thermal convection. 
% We then carry out numerical simulations of magnetohydrodynamic (MHD) dynamos driven by thermal convection. 
% {\color{red}
% Dynamo simulations with a lower Ekman number are necessary, however, we were forced to perform dynamo simulations with $E = 10^{-3}$ to investigate various Rayleigh numbers and inner core radii for computational limitation like dynamo simulations of Lhuillier {\it et al.} \shortcite{Lhuillier:2019}, in which the Ekman number is larger than $E = 10^{-3}$.
% }

%
\section{RESULTS}

\subsection{Estimation of the critical Rayleigh number}

To estimate the critical Rayleigh number, $Ra_{\rm crit}$, equations (\ref{eq:momentum})--(\ref{eq:heat}) without the Lorentz force term $\left(Pm E\right)^{-1} \left(\bvec{J} \times \bvec{B} \right)$, are solved as non-magnetic thermal convection simulations.
The kinetic energy density is calculated for the average of $t / \tau_{\nu} = 4.5$ to 6 in viscous diffusion time as follows: 
%
\begin{equation}
E_{\rm kin} = \frac{1}{V_{S}} \int_{V_{S}} \frac{1}{2} \bvec{u}^{2} dV,
\label{eq:kinetic_energy}
\end{equation}
%
where $V_{S}$ is volume of a spherical shell.

$E_{\rm kin}$ listed in 
{\color{blue} Tables~\ref{table:Rac} and \ref{table:Rac_Ek4}} is calculated as a mean over a quasi-steady state, and is found to be linearly related to, as shown in Fig. \ref{fig:fig_1}.
The critical Rayleigh numbers 
{\color{blue} for $E = 1.0 \times 10^{-3}$}
are estimated as $Ra_{\rm crit} = 1.09 \times 10^5$, $0.72 \times 10^5$, and $0.56 \times 10^5$ for $r_i/r_o = 0.15$, $0.25$, and $0.35$, respectively, 
{\color{blue} and that for $E = 1.0 \times 10^{-4}$ is  $Ra_{\rm crit} = 1.08 \times 10^6$, $0.82 \times 10^6$, and $0.70 \times 10^6$, respectively.
The present estimations are by the same method as Al-Shamali {\it et al.} \shortcite{Al-Shamali:2004}, and}
the obtained values of $Ra_{\rm crit}$ are almost identical to those reported in Al-Shamali {\it et al.} \shortcite{Al-Shamali:2004} for the same parameters and conditions used in this study. 
% $Ra_{\rm crit}$ is large when the aspect ratio is smaller, indicating that convection in a rotating, thick spherical shell requires a large buoyancy.
{\color{blue} In the both $E = 1.0 \times 10^{-3}$ and $1.0 \times 10^{-4}$ cases,}
{\color{red} % MM
$Ra_{\rm crit}$ is larger when the aspect ratio is smaller. These results indicate that convection in a rotating, thicker spherical shell requires a larger buoyancy.
}
%
%
\begin{table}
\caption{Average kinetic energy density $E_{kin}$ at the quasi-steady state \\
for the thermal convection without the magnetic field with $E = 1.0 \times 10^{-3}$.}
\begin{center}
\begin{tabular}{|ccc|ccc|cc|}
   \hline
  \multicolumn{2}{|c|}{$r_{\rm i}/r_{\rm o} = 0.15$} & \hspace{5mm} &
  \multicolumn{2}{|c|}{$r_{\rm i}/r_{\rm o} = 0.25$} & \hspace{5mm} &
  \multicolumn{2}{|c|}{$r_{\rm i}/r_{\rm o} = 0.35$} \\
  $Ra[\times 10^5] $ &  $E_{\rm kin}$ & &
  $Ra[\times 10^5] $ &  $E_{\rm kin}$ & &
  $Ra[\times 10^5] $ &  $E_{\rm kin}$ \\
    \hline
   1.0  &  $8.31 \times 10^{-6} $ & &  0.70 &  $8.55 \times 10^{-4}$ & &  0.55  &  $1.76 \times 10^{-4}$ \\
   1.2  &  5.72 & & 0.75 &  2.41 & &  0.58  &  2.38\\
   1.25 &  8.15 & &  0.78  &  4.94 & &  0.60  &  4.54\\
   1.3  &  10.62 & &  0.80  &  6.62 & &  0.62  &  6.76  \\
   1.35 &  13.16 & &  0.82  & 8.35 & &  0.65  &  10.22 \\
   1.4  &  15.80 & &  0.85  &  11.02 & &  0.67  &  12.60 \\
   1.45  &  18.54 & &  0.90  &  15.70 & &  0.70  &  16.28\\
 \hline
\end{tabular}
\end{center}
\label{table:Rac}
\end{table}

%
\begin{table}
\caption{Average kinetic energy density $E_{kin}$ at the quasi-steady state \\
for the thermal convection without the magnetic field with $E = 1.0 \times 10^{-4}$.}
\begin{center}
\begin{tabular}{|ccc|ccc|cc|}
   \hline
  \multicolumn{2}{|c|}{$r_{\rm i}/r_{\rm o} = 0.15$} & \hspace{5mm} &
  \multicolumn{2}{|c|}{$r_{\rm i}/r_{\rm o} = 0.25$} & \hspace{5mm} &
  \multicolumn{2}{|c|}{$r_{\rm i}/r_{\rm o} = 0.35$} \\
  $Ra[\times 10^6] $ &  $E_{\rm kin}$ & &
  $Ra[\times 10^6] $ &  $E_{\rm kin}$ & &
  $Ra[\times 10^6] $ &  $E_{\rm kin}$ \\
    \hline
  1.15 & 5.70 & & 0.85 & 3.60  & & 0.72 & 3.77 \\
  1.18 &  7.83 & & 0.87 & 5.87 & & 0.73 &  5.19 \\
  1.20 &  9.30 & & 0.89 & 8.20 & & 0.75 &  8.11 \\
  1.23 & 11.56 & & 0.90 & 9.39 & & 0.77 & 11.12 \\
  1.25 & 13.10 & & 0.93 & 13.07 & & 0.78 & 12.65  \\
  1.30 & 17.09 & & 0.95  & 15.61 & & 0.80 & 15.78 \\
 \hline
\end{tabular}
\end{center}
\label{table:Rac_Ek4}
\end{table}


\subsection{Results of MHD simulation}

% We performed MHD dynamo simulations for various Rayleigh numbers and the radius ratios using Eqs (1)--(5).
{\color{red}
We perform MHD dynamo simulations for various combinations of Rayleigh, Ekman numbers and the radius ratios solving Eqs (1)--(5) for at least two magnetic diffusion time to assess whether the magnetic field is sustained or dissipated.
}
The magnetic energy density is calculated as follows:
%
\begin{equation}
E_{\rm mag} = \frac{1}{V_{S}E Pm} \int_{V_{S}} \frac{1}{2} \bvec{B}^{2} dV.
\label{eq:magnetic_energy}
\end{equation}
%
{\color{blue} Tables~\ref{table:Summary_15} to 8 are lists of time averaged results of the present MHD dynamo simulations.
%\ref{table:Summary_3115} Table 8 can not show correctly. Why?
}
%list results of MHD dynamo simulations. 
% We performed respective numerical simulations for at least two magnetic diffusion times to assess whether the magnetic field was sustained or dissipated. 
{\color{blue} For example, Fig.~\ref{fig:fig_2} shows the time evolution of the kinetic and magnetic energy density at $Ra/Ra_{\rm crit} = 21.1$ and $E = 1.0 \times 10^{-4}$ with $r_i/r_o = 0.25$.
The magnetic field is found to be sustained more than the kinetic enegy in this case.
}
We calculate the time average of the kinetic and magnetic energy density, as well as the dipolarity over a 0.5 magnetic diffusion time, at the end of each simulation (see shaded area in Fig.~\ref{fig:fig_2}.
% The kinetic and magnetic energy density as a function of Rayleigh number is shown in Fig.~\ref{fig:fig_3}.
{\color{red} % MM
The kinetic and magnetic energy density as a function of magnetic Reynolds number is shown in Fig.~\ref{fig:fig_3}.
}
%, where the black, red, and blue points are the E_kin values in the non-MHD cases, E_kin values in the MHD cases, and E_mag values in the MHD cases, respectively. The “F” denotes the failed dynamo cases. 
% Reviewer # のコメントに従い削除
In each radius ratio case with a sustained magnetic field, the $E_{\rm kin}$ values in the MHD cases are smaller than the $E_{\rm kin}$ values in the corresponding non-MHD cases. 
These results show that the Lorentz force caused by the intense magnetic field disturbs convection.
There were differences among the three radius ratio cases. 
% At $r_i/r_o = 0.15$, the $E_{\rm mag}$ values in the MHD cases were smaller than the $E_{\rm kin}$ values in the MHD cases for all cases. 
{\color{red} % MM
The $E_{\rm mag}$ values are smaller than the $E_{\rm kin}$ values in all the MHD cases at $r_i / r_o = 0,15$
}
{\color{blue}  %HM
and $E = 1.0 \times 10^{-3}$.
}
This trend is consistent with the results of Heimpel {\it et al.} \shortcite{Heimpel:2005}, whose simulations were performed around dynamo onset. 
At $r_i/r_o = 0.25$ and $E = 1.0 \times 10^{-3}$, the values of $E_{\rm mag}$ in the MHD cases are either smaller or larger than the $E_{\rm kin}$ values in the MHD.
For cases of $Ra/Ra_{\rm crit}  = 3.6$ and $4.0$, the $E_{\rm mag}$ values are significantly smaller than those for the cases of other Rayleigh numbers.
Although the trend in the magnetic energy spectrum did not change, there was a decrease in the amplitude.
% At $r_i/r_o = 0.35$, the values of $E_{\rm mag}$ in the MHD cases were larger than the values of $E_{\rm kin}$ in the MHD cases for almost all cases. 
{\color{blue} At $r_i/r_o = 0.35$ and $E = 10^{-3}$, 
}
{\color{red}
the values of $E_{\rm mag}$ are larger than the values of $E_{\rm kin}$ in almost all the MHD cases.
}
% From the above, it is not likely to sustain a strong magnetic field with a smaller inner core.

{\color{blue}
In the parameter regime with $E = 1.0 \times 10^{-4}$, results have similar characteristics to the results with $E = 1.0 \times 10^{-3}$ but more clearer. At the case with $r_i/r_o = 0.15$ also have a solution with $E_{\rm mag} > E_{\rm kin}$. And, as in the regime with $E = 1.0 \times 10^{-3}$, the range of $Ra / Ra_{\rm crit}$ to sustain $E_{\rm mag} > E_{\rm kin}$ decreases with decreasing the radius ratio $r_i / r_o$.
}
{\color{red}
These results indicate that it is difficult to sustain a strong magnetic field with a smaller inner core.
}
%
\begin{table*}
% \begin{center}
\caption{Time average of the magnetic Reynolds number $Rm$, kinetic energy $E_{\rm kin}$, magnetic energy $E_{\rm mag}$, dipolarity $f_{\rm dip}$, and {\color{red} the conventional and dynamic Elsasser number $\Lambda$ and $\Lambda_{d}$ for the cases with $E = 1.0 \times 10^{-3}$, $Pm = 5.0$, and} $r_{\rm i}/r_{\rm o} = 0.15$}
  \begin{tabular}{ccccccccccc}
      \hline
     $Ra[\times 10^3]$  &  $Ra/Ra_{\rm crit}$&
     {\color{red} $L_{max}$} & {\color{red} $N_{r}$} & {\color{red} $Rm$} & $E_{\rm kin}$  &  $E_{\rm mag}$ & $f_{\rm dip}$ & $f_{\rm mag\_fit}$ & $\Lambda$ & $\Lambda_{\rm d}$\\
    \hline
    $760$  & $7.0$ & 63 & 80 &  202.9 & $823.7$ & $5.193$ & $-$ & $-$ & $-$ & $-$ \\
    $870$  & $8.0$ & 63 & 80 &  200.2 & $801.4$ & $501.6$ & $0.494$ & $1.435$ & 5.016 & $0.105$\\
%    $980$  & $9.0$ & 63 & 80 &  205.7 & $846.37$ & $579.0$ & $0.516$ & $1.866$ & 5.790 & $0.116$\\
%    $1100$  & $10.1$ & 63 & 80 &  193.4 & $748.36$ & $444.7$ & $0.349$ & $0.860$ & 4.447 & $0.053$\\
    $980$  & $9.0$ & 63 & 80 &  205.7 & $846.4$ & $579.0$ & $0.516$ & $1.866$ & 5.790 & $0.116$\\
    $1100$  & $10.1$ & 63 & 80 &  193.4 & $748.4$ & $444.7$ & $0.349$ & $0.860$ & 4.447  & $0.053$\\
    $1300$  & $11.9$ & 63 & 80 &  273.2 & $1493$ & $300.5$ & $0.117$ & $0.322$ & 3.005 & $0.068$\\
    $1500$  & $13.8$ & 63 & 80 &  305.5 & $1867$ & $135.6$ & $0.155$ & $0.391$ & 1.356 & $0.034$\\
    $1700$  & $15.6$ & 63 & 80 & 327.2 & $2141$ & $234.3$ & $0.172$ & $0.420$ & 2.343 & $0.054$\\
     \hline
  \end{tabular}
% \end{center}
 \label{table:Summary_15}
 \end{table*}
 
\begin{table*}
%\begin{center}
\caption{Time average of the magnetic Reynolds number $Rm$, kinetic energy $E_{\rm kin}$, magnetic energy $E_{\rm mag}$, dipolarity $f_{\rm dip}$, and {\color{red} the conventional and dynamic Elsasser number $\Lambda$ and $\Lambda_{d}$ for the cases with $E = 1.0 \times 10^{-3}$, $Pm = 5.0$, and} $r_{\rm i}/r_{\rm o} = 0.25$}
  \begin{tabular}{ccccccccccc}
    \hline
     $Ra[\times 10^3]$  &  $Ra/Ra_{\rm crit}$& 
     {\color{red} $L_{max}$} & {\color{red} $N_{r}$} & {\color{red} $Rm$} & $E_{\rm kin}$  &  $E_{\rm mag}$ & $f_{\rm dip}$ & $f_{\rm mag\_fit}$ & $\Lambda$ & $\Lambda_{\rm d}$\\
    \hline 
    $140$  & $1.9$ & 47 & 60 & 61.92 &  $76.69$ & $-$ & $-$ & $-$ & $-$ & $-$\\
    $160$  & $2.2$ & 47 & 60 & 54.34 &  $59.05$ & $958.6$ & $0.860$ & $2.116$ & 9.586 & $0.355$\\
    $180$  & $2.5$ & 47 & 60 & 56.08 &  $62.89$ & $1097$ & $0.867$ & $2.397$ & 10.97 & $0.410$\\
    $200$  & $2.8$ & 47 & 60 & 65.38 &  $85.49$ & $844.4$ & $0.784$ & $1.828$ & 8.444 & $0.323$\\
    $220$  & $3.1$ & 47 & 60 & 72.08 &  $103.9$ & $769.2$ & $0.757$ & $2.441$ & 7.692 & $0.283$\\
    $260$  & $3.6$ & 47 & 60 & 108.7 &  $236.4$ & $48.4$ & $0.644$ & $2.477$ & 0.484 & $0.021$\\
    $290$  & $4.0$ & 47 & 60 & 120.3 &  $289.3$ & $83.0$ & $0.620$ & $2.610$ & 0.830 & $0.035$\\
    $330$  & $4.6$ & 47 & 60 & 111.5 &  $248.6$ & $753.8$ & $0.602$ & $2.287$ & 7.538 & $0.277$\\
    $360$  & $5.0$ & 47 & 60 & 122.0 &  $297.7$ & $769.0$ & $0.562$ & $1.935$ & 7.690 & $0.224$\\
    $430$  & $6.0$ & 47 & 60 & 150.9 &  $455.6$ & $491.0$ & $0.522$ & $1.887$ & 4.910 & $0.174$\\
    $500$  & $6.9$ & 47 & 60 & 179.2 &  $642.2$ & $305.2$ & $0.456$ & $1.551$ & 3.052 & $0.130$\\
    $580$  & $8.1$ & 47 & 60 & 212.9 &  $906.8$ & $126.0$ & $0.412$ & $1.556$ & 1.260 & $0.059$\\
    $700$  & $9.7$ & 95 & 95 & 246.1 &  1218 & 201.5 & 0.144 & 1.051 & 2.016 & 0.0834 \\
    $900$  & $12.5$ & 95 & 95 & 279.5 &  1568 & 307.2 & 0.0439 & 0.5030 & 3.072 & 0.1294 \\
    \hline
  \end{tabular}
% \end{center}
 \label{table:Summary_25}
\end{table*}
 
\begin{table*}
% \begin{center}
\caption{Time average of the magnetic Reynolds number $Rm$, kinetic energy $E_{\rm kin}$, magnetic energy $E_{\rm mag}$, dipolarity $f_{\rm dip}$, and {\color{red} the conventional and dynamic Elsasser number $\Lambda$ and $\Lambda_{d}$ for the cases with $E = 1.0 \times 10^{-3}$, $Pm = 5.0$, and} $r_{\rm i}/r_{\rm o} = 0.35$}
  \begin{tabular}{ccccccccccc}
    \hline
     $Ra[\times 10^3]$  &  $Ra/Ra_{\rm crit}$& 
     {\color{red} $L_{max}$} & {\color{red} $N_{r}$} & {\color{red} $Rm$} & $E_{\rm kin}$  &  $E_{\rm mag}$ & $f_{\rm dip}$ & $f_{\rm mag\_fit}$ & $\Lambda$ & $\Lambda_{\rm d}$\\
    \hline
      $84$  & $1.5$ & 47 & 60 & 41.89 &   $35.09$ & $-$ & $-$ & $-$ & $-$ & $-$ \\
     $110$  & $2.0$ & 47 & 60 & 46.70 &  $43.61$ & $819.6$ & $0.816$ & $4.713$ & 8.196 & $0.420$\\
     $140$  & $2.5$ & 47 & 60 & 66.84 &  $89.35$ & $1408$ & $0.724$ & $3.174$ & 14.08 & $0.519$\\
     $170$  & $3.0$ & 47 & 60 & 73.08 &  $106.8$ & $950.2$ & $0.739$ & $4.239$ & 9.502 & $0.407$\\
     $200$  & $3.6$ & 47 & 60 & 82.55 &  $136.3$ & $890.2$ & $0.692$ & $3.900$ & 8.902 & $0.399$\\
     $230$  & $4.1$ & 47 & 60 & 98.26 &  $193.1$ & $938.4$ & $0.632$ & $2.946$ & 9.384 & $0.421$\\
     $280$  & $5.0$ & 47 & 60 & 124.9 &  $311.9$ & $895.6$ & $0.556$ & $2.848$ & 8.956 & $0.383$\\
     $340$  & $6.1$ & 47 & 60 & 159.5 &  $508.7$ & $713.6$ & $0.480$ & $2.006$ & 7.136 & $0.294$\\
     $400$  & $7.1$ & 47 & 60 & 204.6 &  $837.6$ & $73.46$ & $0.376$ & $2.071$ & 0.7346 & $0.035$\\
     $450$  & $8.0$ & 47 & 60 & 230.2 &  $1060$ & $-$ & $-$ & $-$ & $-$ & $-$ \\
     $700$  & $12.5$ & 95 & 95 & 307.8 & $1902$ & $372.0$ & 0.02923 & 0.4508 & 3.720 & 0.1429 \\
    \hline
  \end{tabular}
% \end{center}
\label{table:Summary_35}
 \end{table*}
%
\begin{table*}
\label{table:Summary_3115}
 \end{table*}

%
\begin{table*}
% \begin{center}
{\color{red}
\caption{Time average of the magnetic Reynolds number $Rm$, kinetic energy $E_{\rm kin}$, magnetic energy $E_{\rm mag}$, dipolarity $f_{\rm dip}$, and  the conventional and dynamic Elsasser number $\Lambda$ and $\Lambda_{d}$ for the cases with $E = 1.0 \times 10^{-4}$, $Pm = 2.0$, and $r_{\rm i}/r_{\rm o} = 0.15$.}
}
{\color{red}
  \begin{tabular}{ccccccccccc}
    \hline
     $Ra[\times 10^6]$  &  $Ra/Ra_{\rm crit}$& 
     $L_{max}$ & $N_{r}$ & $Rm$ 
     & $E_{\rm kin}$  &  $E_{\rm mag}$ & $f_{\rm dip}$ & $f_{\rm mag\_fit}$ & $\Lambda$ & $\Lambda_{\rm d}$\\
    \hline
      3.824 & 3.551 & 95 & 95 & 67.36 & 567.2 & $-$ & $-$ & $-$ & $-$ & $-$ \\
      7.647 & 7.103 & 95 & 95 & 106.3 & 1422 & $-$ & $-$ & $-$ & $-$ & $-$ \\
      9.176 & 8.523 & 95 & 95 & 118.9 & 1780 & 2277 & 0.7032 & 25.99 & 0.9109 & 0.02205 \\
      11.47 & 10.65 & 95 & 95 & 132.8 & 2221 & 5398 & 0.7358 & 31.95 & 2.159 & 0.09387 \\
      15.29 & 14.21 & 95 & 95 & 161.0 & 3260 & 9280 & 0.6910 & 27.39 & 3.712 & 0.1425 \\
      26.76 & 24.86 & 95 & 95 & 252.2 & 7994 & 12906 & 0.4801 & 12.24 & 5.162 & 0.1739 \\
      38.23 & 35.51 & 95 & 95 & 338.9 & 14424 & 12192 & 0.3553 & 9.387 & 4.877 & 0.1583 \\
      61.18 & 56.82 & 95 & 95 & 501.9 & 31610 & 10306 & 0.02114 & 0.6084 & 4.123 & 0.1231 \\
   \hline
  \end{tabular}
 }
% \end{center}
\label{table:Summary_415}
\end{table*}
%
%
%
%
\begin{table*}
% \begin{center}
{\color{red}
\caption{Time average of the magnetic Reynolds number $Rm$, kinetic energy $E_{\rm kin}$, magnetic energy $E_{\rm mag}$, dipolarity $f_{\rm dip}$, and  the conventional and dynamic Elsasser number $\Lambda$ and $\Lambda_{d}$ for the cases with $E = 1.0 \times 10^{-4}$, $Pm = 2.0$, and $r_{\rm i}/r_{\rm o} = 0.25$.}
}
{\color{red}
  \begin{tabular}{ccccccccccc}
    \hline
     $Ra[\times 10^6]$  &  $Ra/Ra_{\rm crit}$& 
     $L_{max}$ & $N_{r}$ & $Rm$ 
     & $E_{\rm kin}$  &  $E_{\rm mag}$ & $f_{\rm dip}$ & $f_{\rm mag\_fit}$ & $\Lambda$ & $\Lambda_{\rm d}$\\
    \hline
      2.60 & 3.551 & 95 & 95 & 51.99 & 338.5 & $-$ & $-$ & $-$ & $-$ & $-$ \\
      433.3 & 5.278 & 95 & 95 & 89.68 & 1408 & 1360 & 0.8091 & 32.87 & 0.5633 & 0.03626 \\
      650 & 7.917 & 95 & 95 & 118.9 & 1774 & 6706 & 0.7118 & 26.39 & 2.683 & 0.2260 \\
      866.7 & 10.56 & 95 & 95 & 141.6 & 2518 & 12212 & 0.6980 & 33.44 & 4.885 & 0.2746 \\
      1733 & 21.11 & 95 & 95 & 269.4 & 9101 & 18615 & 0.4207 & 12.41 & 7.446 & 0.2746 \\
      3467 & 42.22 & 95 & 95 & 543.0 & 36954 & 11862 & 0.02337 & 0.9429 & 4.745 & 0.1560 \\
    \hline
  \end{tabular}
 }
% \end{center}
\label{table:Summary_25_Ek4}
\end{table*}
%
%
\begin{table*}
% \begin{center}
{\color{red}
\caption{Time average of the magnetic Reynolds number $Rm$, kinetic energy $E_{\rm kin}$, magnetic energy $E_{\rm mag}$, dipolarity $f_{\rm dip}$, and  the conventional and dynamic Elsasser number $\Lambda$ and $\Lambda_{d}$ for the cases with $E = 1.0 \times 10^{-4}$, $Pm = 2.0$, and $r_{\rm i}/r_{\rm o} = 0.35$.}
}
{\color{red}
  \begin{tabular}{ccccccccccc}
    \hline
     $Ra[\times 10^6]$  &  $Ra/Ra_{\rm crit}$& 
     $L_{max}$ & $N_{r}$ & $Rm$ 
     & $E_{\rm kin}$  &  $E_{\rm mag}$ & $f_{\rm dip}$ & $f_{\rm mag\_fit}$ & $\Lambda$ & $\Lambda_{\rm d}$\\
    \hline
      2.0 & 2.876 & 95 & 95 & 52.28 & 341.7 & $-$ & $-$ & $-$ & $-$ & $-$ \\
      3.0 & 4.314 & 95 & 95 & 79.06 & 783.2 & 1485 & 0.7507 & 3.740 & 0.5939 & 0.05238 \\
      4.0 & 5.752 & 95 & 95 & 96.65 & 1174 & 5307 & 0.7429 & 29.59 & 2.123 & 0.1581 \\
      5.0 & 7.190 & 95 & 95 & 106.4 & 1421 & 11133 & 0.7566 & 41.54 & 4.453 & 0.2840 \\
      7.5 & 10.79 & 95 & 95 & 161.9 & 3294 & 18096 & 0.5862 & 23.89 & 7.238 & 0.3724 \\
      10.0 & 14.38 & 95 & 95 & 222.8 & 6226 & 19662 & 0.4417 & 13.352 & 7.865 & 0.3671 \\
      15.0 & 21.57 & 95 & 95 & 328.4 & 13481 & 22715 & 0.3138 & 9.896 & 9.086 & 0.3690  \\
      20.0 & 28.76 & 95 & 95 & 503.0 & 31679 & 10105 & 0.01102 & 0.6228 & 4.042 & 0.1651 \\
    \hline
  \end{tabular}
 }
% \end{center}
\label{table:SummaryEk4_35}
\end{table*}
%
%
%
%
% Fig.~\ref{fig:Snap_non_dipoler_E1-3} shows the radial component of the magnetic field at the CMB and the equatorial cross-sections of the $z$-component of the vorticity and magnetic field are plotted at $Ra/Ra_{\rm crit} = 11.9$ for $r_i/r_o = 0.15$. 

{\color{blue}
Now, we compare characteristics of the field structures among cases with intense magnetic field and with weak magnetic field. We plot the temperature, $z$-comonent of the vorticity $\omega_{z}$ and magnetic field $B_{z}$, and radial magnetic field $B_{r}$ at CMB for the case with the weak and strong magnetic field with $E = 1.0 \times 10^{-3}$ in Figures \ref{fig:Snap_non_dipoler_E1-3} and \ref{fig:Snap_Dypoler_E1-3}, respectively. And the same plots with weak and strong magnetic fields int the regime $E = 1.0 \times 10^{-4}$ are shown in Figures \ref{fig:Snap_non_dipoler_E1-4} and \ref{fig:Snap_Dypoler_E1-4}
}
%{\color{red} % MM
%Fig.~\ref{fig:Snap_non_dipoler_E1-3} shows the radial component of the magnetic field at the CMB, and the $z$-component of the vorticity and magnetic field on the equatorial cross-sections at $Ra/Ra_{\rm crit} = 11.9$ for $r_i/r_o = 0.15$. 
%}
%The same plots at $Ra/Ra_{\rm crit} = 3.1$ for $r_i/r_o = 0.25$ and at $Ra/Ra_{\rm crit} = 3.0$ for $r_i/r_o = 0.35$ are shown in Figs~\ref{fig:fig_5} and \ref{fig:fig_6}, respectively. 
% At the equatorial plane, the magnetic field is concentrated in the anti-cyclone columns to generate a dipolar field at $r_i/r_o = 0.25$ and $0.35$; intense magnetic patches are located near the tangent cylinder, which is an imaginary cylinder tangent to the inner-core equator and coaxial with the rotation axis. 
{\color{blue}
The magnetic field on the equatorial plane is concentrated in the anti-cyclonic columns to generate a dipolar field at cases with large dipolarity cases in Figs.~\ref{fig:Snap_Dypoler_E1-3} and \ref{fig:Snap_Dypoler_E1-4}; intense magnetic patches are located near the tangent cylinder, which is an imaginary cylinder tangent to the inner-core at the equator and coaxial with the rotation axis. 

In the cases with small dipolarity in Figs.~\ref{fig:Snap_non_dipoler_E1-3} and \ref{fig:Snap_non_dipoler_E1-4},
}
strong convection is generated locally, where strong $B_{z}$ convection is generated between the cyclonic and anti-cyclonic columns at the equatorial plane.
As these intense magnetic fields are not concentrated in the convection columns, the radial magnetic field at the CMB near the tangent cylinder has a quadrupolar (symmetric with respect to the equator) and is smaller than that of the cases with different aspect ratios.

{\color{blue}
Focusing on the difference of the aspect ratio $r_i/r_o$, same characteristics can be found in Figs.~\ref{fig:Snap_non_dipoler_E1-3} to \ref{fig:Snap_Dypoler_E1-4}. Number of hot upward ({\it i.e.} high temperature region) around the inner boundary decreases with small $r_i/r_o$. As a results, intense convection columns is more localized with smaller $r_i/r_o$.
}
%
%
%
%
\begin{figure}
\begin{center}
\[
\begin{array}{cc}
\mbox{$E = 1.0 \times 10^{-3}$} & \mbox{$E = 1.0 \times 10^{-4}$} \\
\includegraphics*[width=70mm]{Figures/Rac_Ek3.pdf} &
\includegraphics*[width=70mm]{Figures/Rac_Ek4.pdf}
\end{array}
\]
\end{center}
\caption{{\color{red}
The kinetic energy density averaged over the spherical shell as a function of the Rayleigh number with different geometries. Results for $E = 1.0 \times 10^{-3}$ and $1.0 \times 10^{-4}$ are plotted in the left and right panel, respectively. Results for each run is plotted by marks, and linear fitting of the results are drown by lines.  Results for $r_{i} / r_{o}= 0.15$, 0.25, and 0.35 are colored by Red, green, and blue, respectively.}
}
\label{fig:fig_1}
\end{figure}
%
\begin{figure}
\begin{center}
\[
\begin{array}{c}
\includegraphics*[width=80mm]{Figures/sph_shell_383_ene.pdf}
%\includegraphics*[width=70mm]{Figures/sph_shell_378_ene.pdf}
\end{array}
%\includegraphics*[width=80mm]{Figures/Fig2.png}
\]
\end{center}
%\caption{Time evolution of the kinetic and magnetic energy densities in the case of sustained dynamo at $Ra / Ra_{\rm crit} = 2.8$ with $r_{i} / r_{o} = 0.25$ . The red and blue lines mean the kinetic and magnetic energy density, respectively.
%}
\caption{
\color{blue} Time evolution of the kinetic and magnetic energy densities in the case of sustained dynamo at $E = 1.0 \times 10^{-4}$, $Ra / Ra_{\rm crit} = 21.1$ with $r_{i} / r_{o} = 0.25$. The red and blue lines mean the kinetic and magnetic energy density, respectively. The time average is taken at the end of each simulation for 0.5 times of the magnetic diffusion time which is shown by gray.
}
\label{fig:fig_2}
\end{figure}
%
%
\begin{figure*}
\begin{center}
\[
\begin{array}{ccc}
\multicolumn{3}{c}{\mbox{$E = 1.0 \times 10^{-3}$}} \\
\multicolumn{3}{c}{\includegraphics*[width=150mm]{Figures/Fig3.png}} \\
\multicolumn{3}{c}{\mbox{$E = 1.0 \times 10^{-4}$}} \\
\includegraphics*[width=50mm]{Figures/Ene_vs_RaRatio_015_E4.pdf} &
\includegraphics*[width=50mm]{Figures/Ene_vs_RaRatio_025_E4.pdf} &
\includegraphics*[width=50mm]{Figures/Ene_vs_RaRatio_035_E4.pdf}
\end{array}
\]
\end{center}
\caption{The kinetic and magnetic energy density as a function of the ratio of Rayleigh number to the critical Rayleigh number in spherical shells with different geometries. The black, red, and blue points are the $E_{\rm kin}$ values in the non-MHD cases, $E_{\rm kin}$ values in the MHD cases, and $E_{\rm mag}$ values in the MHD cases, respectively. The “F” denotes the failed dynamo cases.
}
\label{fig:fig_3}
\end{figure*}
%
%
\begin{figure*}
\begin{center}
\[
\begin{array}{c}
 \mbox{$Ra / Ra_{\rm crit} = 11.9$, $r_{i} / r_{o} = 0.15$} \\
 \includegraphics*[width=150mm]{Figures/A15_image_4.png} \\
  \mbox{$Ra / Ra_{\rm crit} = 9.722$, $r_{i} / r_{o} = 0.25$} \\
\includegraphics*[width=150mm]{Figures/sph_shell_456_300.png} \\
  \mbox{$Ra / Ra_{\rm crit} = 12.5$, $r_{i} / r_{o} = 0.35$} \\
 \includegraphics*[width=150mm]{Figures/sph_shell_471_100.png}
\end{array}
%\includegraphics*[width=160mm]{Figures/A15_image_4}
\]
\end{center}
\caption{Spatial pattern of the temperature, flow,  and magnetic fields for the cases with non-dipolar solution and $E = 1.0 \times 10^{-3}$. The temperature (a), $z$-component of the vorticity $\omega_{z}$ (b) and magnetic field $B_{z}$ (c) at the equatorial plane, and the radial magnetic field $B_{r}$ at the CMB (d) are plotted, respectively.
}
\label{fig:Snap_non_dipoler_E1-3}
\end{figure*}
%
%
\begin{figure*}
\begin{center}
\[
\begin{array}{c}
 \mbox{$Ra / Ra_{\rm crit} = 3.055$, $r_{i} / r_{o} = 0.25$} \\
 \includegraphics*[width=150mm]{Figures/A25_image_4.png} \\
 \mbox{$Ra / Ra_{\rm crit} = 3.036$, $r_{i} / r_{o} = 0.35$} \\
 \includegraphics*[width=150mm]{Figures/A35_image_4.png}
\end{array}
%\includegraphics*[width=160mm]{Figures/Fig5.png}
\]
\end{center}
\caption{Spatial pattern of the temperature, flow,  and magnetic fields for the cases with dipolar solution and $E = 1.0 \times 10^{-3}$. The temperature (a), $z$-component of the vorticity $\omega_{z}$ (b) and magnetic field $B_{z}$ (c) at the equatorial plane, and the radial magnetic field $B_{r}$ at the CMB (d) are plotted, respectively.
}
\label{fig:Snap_Dypoler_E1-3}
\end{figure*}
%
%
\begin{figure*}
\begin{center}
\[
\begin{array}{c}
\begin{array}{c}
 \mbox{$Ra / Ra_{\rm crit} = 56.82$, $r_{i} / r_{o} = 0.15$} \\
 \includegraphics*[width=150mm]{Figures/sph_shell_376_2963.png} \\
 \mbox{$Ra / Ra_{\rm crit} = 42.22$, $r_{i} / r_{o} = 0.25$} \\
 \includegraphics*[width=150mm]{Figures/sph_shell_384_4800.png} \\
 \mbox{$Ra / Ra_{\rm crit} = 28.76$, $r_{i} / r_{o} = 0.35$} \\
 \includegraphics*[width=150mm]{Figures/sph_shell_393_2384.png}
\end{array}
%\includegraphics*[width=160mm]{Figures/Fig6.png}
\end{array}
\]
\end{center}
\caption{Spatial pattern of the temperature, flow,  and magnetic fields for the cases with non-dipolar solution and $E = 1.0 \times 10^{-4}$. The temperature (a), $z$-component of the vorticity $\omega_{z}$ (b) and magnetic field $B_{z}$ (c) at the equatorial plane, and the radial magnetic field $B_{r}$ at the CMB (d) are plotted, respectively.
}
\label{fig:Snap_non_dipoler_E1-4}
\end{figure*}
%
%
\begin{figure*}
\begin{center}
\[
\begin{array}{c}
  \mbox{$Ra / Ra_{\rm crit} = 10.65$, $r_{i} / r_{o} = 0.15$} \\
\includegraphics*[width=150mm]{Figures/sph_shell_377_1250.png} \\
 \mbox{$Ra / Ra_{\rm crit} = 7.917$, $r_{i} / r_{o} = 0.25$} \\
 \includegraphics*[width=150mm]{Figures/sph_shell_387_1000.png} \\
 \mbox{$Ra / Ra_{\rm crit} = 7.190$, $r_{i} / r_{o} = 0.35$} \\
 \includegraphics*[width=150mm]{Figures/sph_shell_495_500.png}
\end{array}
%\includegraphics*[width=160mm]{Figures/Fig6.png}
\]
\end{center}
\caption{Spatial pattern of the temperature, flow,  and magnetic fields for the cases with dipolar solution and $E = 1.0 \times 10^{-4}$. The temperature (a), $z$-component of the vorticity $\omega_{z}$ (b) and magnetic field $B_{z}$ (c) at the equatorial plane, and the radial magnetic field $B_{r}$ at the CMB (d) are plotted, respectively.
}
\label{fig:Snap_Dypoler_E1-4}
\end{figure*}
%
%

%
\section{DISCUSSION}
% \subsection{Quantitative investigation of dipole dominancy}
{\color{red}
\subsection{Another index of dipolarity}
}
\label{subsec:dipolarity}
\begin{figure}
\begin{center}

\includegraphics*[width=80mm]{Figures/Fig7_new.png}

\end{center}
\caption{
%Magnetic energy density at the CMB of simulation data at $l = 1$ and fitting value of $l = 1$ as a function of spherical harmonic odd degrees in the case of $Ra / Ra_{\rm crit} = 2.8$ and $r_{i} / r_{o} = 0.25$ in  the regime with $E = 1.0 \times 10^{-3}$.
Magnetic energy spectrum at the CMB obtained from numerical simulation data at $E = 10^{-3}$ and $Ra / Ra_{\rm crit} = 2.8$ for $r_i / r_o = 0.25$ as a function of spherical harmonic odd degree, $l$.
The ratio of $E_{\rm mag\_data}^{l=1}$ to $E_{\rm mag\_fitting}^{l=1}$ is another index of dipolarity, $f_{\rm mag\_fit}$.
}
\label{fig:mag_fit_example}
\end{figure}



\begin{figure}
\begin{center}
\[
\includegraphics*[width=160mm]{Figures/Fig10_mod.png}
\]
\end{center}
\caption{
%Magnetic energy density at the CMB of simulation data at $l = 1$ and fitting value of $l = 1$ as a function of spherical harmonic odd degrees in the case of $E = 1.0 \times 10^{-3}$. 
Magnetic energy sectra at the CMB obtained from numerical simulation data as a function of spherical harmonic odd degree, $l$.
(a) A dipolar case at $E = 10^{-3}$ and $Ra / Ra_{\rm crit} = 7.1$ for $r_{i} / r_{o} = 0.35$, and (b) a multipolar case at $E = 10^{-3}$ and $Ra / Ra_{\rm crit} = 10.1$ for $r_{i} / r_{o} = 0.15$.
}
\label{fig:mag_fitting}
\end{figure}



% Although dipolarity has been evaluated in some numerical dynamos \cite{Uli:2006,Soderlund:2012}, it is not sufficiently valid in dynamos whose dipolarities are gradually changing (e.g., Aubert {\it et al.} 2009).
{\color{red}
The dipolarity $f_{\rm dip}$ has been evaluated in numerical dynamos to assess dominamce of the axial dipole magnetic field \cite{Uli:2006,Soderlund:2012}.
The threshold for the dominance of the dipole component, $f_{\rm dip} = 0.35$, was determined from results of many previous numerical dynamos.
One of characteristic behaviours of the palaeomagnetic field can be given by the dipolarity in the range between 0.34 and 0.56 \cite{Meduri:2021}.
However, it is not sufficiently valid in dynamos whose dipolarities are gradually changing (e.g., Aubert {\it et al.} 2009).
% In an observational magnetic field, the dipole is assessed by how far the dipolar component is from the trend of higher degree components \cite{Lowes:1974,Langel:1982}. 
The Earth's dipole magnetic field is also assessed by how far the dipole component is from the trend of higher degree components in the magnetic energy spectrum \cite{Lowes:1974,Langel:1982}.
% We quantitatively evaluate dipolar component dominance in combination with the dipolarity, comparison of the dipolar magnetic energy, and an extrapolation of $l = 1$ based on the fitting curve of higher degrees.
Here we quantitatively evaluate 
%the dominance of the dipole component in combination with the dipolarity, $f_{\rm dip}$, defined in eq.~(\ref{eq:f_dip}), and 
another index, $f_{\rm mag\_fit}$, the ratio of magnetic energy at the CMB for the dipole component to that obtained from extrapolation of the magnetic energy spectrum for the non-dipole components.
% 以下は上記の繰り返しなので削除
% The dipole component of the geomagnetic field is assessed by its power spectrum \cite{Lowes:1974,Langel:1982}.
}
%{\color{blue}
%As shown in Fig.~\ref{fig:fdip_vs_Racratio} in the previous section, 
%}
% The results of many previous numerical dynamo simulations determine the threshold of the dipolar dominance for $f_{\rm dip} = 0.35$. 
{\color{red}
% To obtain a clearer threshold for the dipole dominance, we focused on the magnetic energy spectrum at the CMB as a function of the spherical harmonic degree, $l$. 
In a way similar to the mentioned above but a sophisticated way, to obtain a clearer threshold for the dipolar dominance, we investigate the magnetic energy spectrum at the CMB as a function of the spherical harmonic degree, $l$ (Fig.~\ref{fig:mag_fit_example}).
% at $E = 1.0 \times 10^{-3}$ and $Ra / Ra_{\rm crit} = 7.1$ for $r_i / r_o = 0.35$.
%%% The above is given in the figure caption.
% For example, Fig.~\ref{fig:mag_fitting} shows the magnetic energy density as a function of the spherical harmonic degree at $Ra/Ra_{\rm crit} = 2.8$ for $r_i/r_o = 0.25$. 
% Using odd-degree components in the magnetic energy from $l = 3$ to $19$, we evaluated the fitting curve as $46.21 \times 1.481^{-l}$. 
We obtain a fitting curve as $46.21 \times 1.481^{-l}$ using odd-degree components from $l = 3$ to $19$ in the magnetic energy spectrum.
% At degree $l = 1$, the $E_{\rm mag}$ of the simulation data $E_{\rm mag\_data}^{l = 1}$ was compared with that from the extrapolated value in the fitting function $E_{\rm mag\_fitting}^{l = 1}$. 
% Then, we acquired the ratio of $E_{\rm mag}$ for the simulation result to that from the extrapolated value, $E_{\rm mag\_data}^{l = 1}/E_{\rm mag\_fitting}^{l=1}$ (hereafter referred to as $f_{\rm mag\_fit}$. 
Then we derive the ratio of $E_{\rm mag\_data}^{l=1}$, the magnetic energy for the dipole component evaluated from data of the numerical simulation, to $E_{\rm mag\_fitting}^{l=1}$, the magnetic energy at $l=1$ evaluated from the extrapolation of the fitting curve, as 
%
\begin{equation}
f_{\rm mag\_fit} = \frac{E_{\rm mag\_data}^{l=1} }{E_{\rm mag\_fitting}^{l=1}} .
\label{eq:def_f_mag_fit}
\end{equation}
%
% We can assess the dipolar component dominance from a higher degree trend based on how much the ratio of the extrapolation from fitting $f_{\rm mag\_fit}$ is larger than $1$.
We can judge that the magnetic field in the numerical dynamo is in the regime of dipolar dominance when the value of $f_{\rm mag\_fit}$ is larger than $1$.
}

% {\color{blue}
% Comparing the result between $f_{dip}$ and $f_{\rm mag\_fit}$ in Fig.~\ref{fig:fdip_fit_Eratio}, the common point is that the dipolar dominance decreases with an increasing the magnetic Reynolds number ({\it i.e.} Rayleigh number). 
% In addition, The dependence on the radius ratio is very small by referring the magnetic Reynolds number instead of the Rayleigh number.
% In the regime with $E = 1.0 \times 10^{-3}$, the dipolar non-dominance can be represented by approximately $Rm > 200$ for the all radius ratio cases. 
% In the regime with $E = 1.0 \times 10^{-4}$, the dipolar non-dominance can be represented by approximately $Rm > 400$. 
% As seen in the bottom panels of Fig. \ref{fig:fdip_fit_Eratio}, the ratio of the magnetic to kinetic energies as a function of $Rm$ has also small dependence on the radius ratio in the large $Rm$. 
% These results suggest that the upper limit of to sustain the dipolar magnetic field is not controlled by the geometry of the spherical shell.
% }

{\color{green}
There are a few cases in which it is difficult to categorise the dynamo regime into a dipolar dominated regime or a multipolar dominated one on the basis of a value of $f_{\rm dip}$ only or a value of $f_{\rm mag\_fit}$ only.
Fig.~\ref{fig:mag_fitting}(b) shows an example, in which the dipole component is not dominant as found from $f_{\rm mag\_fit} = 0.860$, although $f_{\rm dip} = 0.349$ is nearly equal to the threshold value, 0.35.
Another example is found in Table~\ref{table:Summary_25} for $E = 10^{-3}$, $Ra / Ra_{\rm crit} = 9.7$, and $r_i / r_o = 0.25$.
In this case, $f_{\rm mag\_fit} = 1.051$ nearly equal to the threshold value 1.0 suggests that the dipole component can be dominant.
However, $f_{\rm dip} = 0.144$ is significantly smaller than the threshold value, 0.35, and the dynamo is found to be in the multipolar dominated regime, as shown in Fig.~\ref{fig:Snap_non_dipoler_E1-3}.
Thus, if a value of $f_{\rm dip}$ or $f_{\rm mag\_fit}$ is close to its threshold value, it is needed to refer to the other index to judge the dynamo regime.
}

\begin{figure}
\begin{center}
\[
\begin{array}{cc}
\mbox{$E = 10^{-3}$} &
\mbox{$E = 10^{-4}$} \\
\includegraphics*[width=75mm]{Figures/fdip_vs_Rm_Ek3.pdf} &
\includegraphics*[width=75mm]{Figures/fdip_vs_Rm_Ek4.pdf} \\
\includegraphics*[width=75mm]{Figures/ffit_vs_Rm_Ek3.pdf} &
\includegraphics*[width=75mm]{Figures/ffit_vs_Rm_Ek4.pdf} \\
\includegraphics*[width=75mm]{Figures/Eratio_vs_Rm_Ek3.pdf} &
\includegraphics*[width=75mm]{Figures/Eratio_vs_Rm_Ek4.pdf}
\end{array}
%\includegraphics*[width=80mm]{Figures/Fig9.png}
\]
\end{center}
\caption{The dipolarity $f_{\rm dip}$, the ratio of the dipolar magnetic energy density at the CMB to the extrapolation of the spherical harmonic degree $l = 1$, $f_{\rm mag\_fit}$, and the ratio of the magnetic to kinetic energy densities $E_{\rm mag} / E_{\rm kin}$ averaged over the spherical shells as a function of the magnetic Reynolds numnber $Rm$ for different geometries. $f_{\rm dip}$,  $f_{\rm mag\_fit}$, and $E_{\rm mag} / E_{\rm kin}$ are shown in the top, middle, and bottom panels, respectively. The settings for $E = 10^{-3}$ and $E = 10^{-4}$ are shown in the left and right panels, respectively. The blue, green and red symbols indicate the cases of $r_{i} / r_{o} = 0.15$, $0.25$ and $0.35$, respectively. 
}
\label{fig:fdip_fit_Eratio}
\end{figure}

% \subsection{Dependency of dynamo regimes on the magnetic Reynolds number}
\subsection{Dynamo regime in relation to inner core size}
\label{subsec:dynamo_regime}
{\color{red}
Fig.~\ref{fig:fdip_fit_Eratio} shows $f_{\rm dip}$, $f_{\rm mag\_fit}$ and $E_{\rm mag} / E_{\rm kin}$ as a function of the magnetic Reynolds number, $Rm$, for $E = 10^{-3}$ and $E = 10^{-4}$.
As a whole, $f_{\rm dip}$, $f_{\rm mag\_fit}$ and $E_{\rm mag} / E_{\rm kin}$ in the case of $E = 10^{-3}$ decrease with increase of $Rm$.
As for those in the case of $E = 10^{-4}$, they once increase and decrease with increase of $Rm$.
The decrease of $f_{\rm dip}$, $f_{\rm mag\_fit}$ and $E_{\rm mag} / E_{\rm kin}$ in the range of large $Rm$ seems to be a common characteristics irrespective of the radius ratio.
The threshold value of the magnetic Reynolds number between dipolar and multidipolar dynamos is found to be $Rm \approx 200$ for $E = 10^{-3}$ and $Rm \approx 400$ for $E = 10^{-4}$.
These results suggest that the upper limit of $Rm$ to sustain a strong dipole magnetic field is not affected by the geometry of the spherical shell, $r_i / r_o$.
}

%Here, $f_{\rm mag\_fit}$ was calculated in all cases and plotted as a function of the Rayleigh number at $r_i/r_o = 0.15$, $0.25$, and $0.35$ (Fig.~\ref{fig:fdip_fit_Eratio}). 
%At $r_i/r_o = 0.15$, $f_{\rm mag\_fit}$ was smaller than $1$ at $Ra/Ra_{\rm crit} > 10.1$. 
%At $r_i/r_o = 0.25$, $f_{\rm mag\_fit}$ was approximately $2.1$ at $Ra/Ra_{\rm crit} = 2.2$ and gradually decreased to $1.6$ with increase of $Ra/Ra_{\rm crit}$ of up to approximately $8.0$. 
%At $r_i/r_o = 0.35$, $f_{\rm mag\_fit}$ was approximately $4.7$ at $Ra/Ra_{\rm crit} = 2.0$, gradually decreasing to $2.1$ with increase of $Ra/Ra_{\rm crit}$ of up to approximately $7.0$.

%Comparing the result between $f_{dip}$ and $f_{\rm mag\_fit}$ in Fig. \ref{fig:fdip_fit_Eratio}, the common point is that the dipolar dominance decreases with an increasing Rayleigh number. 
%At $r_i/r_o = 0.15$, the dipolar non-dominance can be represented by $Ra/Ra_{\rm crit} > 10.1$ for both indices. 
%At $r_i/r_o = 0.25$ and $0.35$, the magnitude relationship of $f_{\rm dip}$ and $f_{\rm mag\_fit}$ is reversed at a small $Ra/Ra_{\rm crit}$ value. 
%Because the magnetic energy of higher degrees is relatively larger for the total energy at $r_i/r_o = 0.35$ than at $r_i/r_o = 0.25$, $f_{\rm dip}$ is smaller at $r_i/r_o = 0.35$ than at $r_i/r_o = 0.25$. 
%Although $E_{\rm mag\_fitting}^{l = 1}$ is almost the same at $r_i/r_o = 0.25$ and $0.35$, $E_{\rm mag\_data}^{l = 1}$ is significantly larger, such that $f_{\rm mag\_fit}$ is larger at $r_i/r_o = 0.35$ than at $r_i/r_o = 0.25$. 
%The difference between $f_{\rm dip}$ and $f_{\rm mag\_fit}$ derives from whether a higher-degree spectrum is taken into account or not. 
%The dependency of the dipolar dominance on the radius ratio can be revealed by $f_{\rm mag\_fit}$ as it contains information from a higher degree spectrum.

{\color{blue}
% We can also show some cases in which we could not categorize the dipole or non-dipole based only on the dipolarity because $f_{\rm mag\_fit}$ is more than 1.0 while $f_{\rm dip}$ is less than 0.35 or {\it vis versa}.
% We can show some cases in which it is difficult to categorise the dynamo regime into a dipolar dominated regime or a multipolar dominated one on the basis of the dipolarity, $f_{\rm dip}$, only.
% In these cases, $f_{\rm mag\_fit}$ was found to be more than unity even if $f_{\rm dip}$ is less than 0.35.
% In the Regime $E = 1.0 \times 10^{-3}$, The magnetic spectra at CMB for the cases for $f_{\rm dip} = 0.376$ at $Ra/Ra_{\rm crit} = 7.1$ with $r_i/r_o = 0.35$ and $f_{\rm dip} = 0.349$ at $Ra/Ra_{\rm crit} = 10.1$ for $r_i/r_o = 0.15$ are shown in Fig.~\ref{fig:mag_fitting} as examples.
}
%We show an example in which we could not categorize the dipole or non-dipole based only on the dipolarity, {\it i.e.}, the cases for $f_{\rm dip} = 0.376$ at $Ra/Ra_{\rm crit} = 7.1$ with $r_i/r_o = 0.35$ and $f_{\rm dip} = 0.349$ at $Ra/Ra_{\rm crit} = 10.1$ for $r_i/r_o = 0.15$. 
%Fig. \ref{fig:mag_fitting} shows the CMB spectra for these two cases.
% The dipolar component is dominant against the high degree trend in the former case while it is not dominant in the latter case. 
% The ratio of extrapolation from the fitting is $f_{\rm mag\_fit} = 2.071$ in the former case and $f_{\rm mag\_fit} = 0.860$ in the latter case; we observed that the former case is dipolar-dominated while the latter case is non-dipolar dominated. 
% The results obtained for $f_{\rm mag\_fit}$  also indicate the dependence of the dipolar dominance on the inner core size. 
% The dipolar dominance becomes weaker with a smaller inner core by calculating the dipolar magnetic energy at the CMB \cite{Heimpel:2005}. 
% In this study, although this tendency was not observed from the dipolarity, it was clear based on the ratio of extrapolation from fitting.
% {\color{red}
% Fig.~\ref{fig:mag_fitting}(b) shows an example, in which the dipole component is not dominant, although we obtained $f_{\rm dip} = 0.349$, which is nearly equal to the threshold value, 0.35.
% On the other hand, we obtained $f_{\rm mag\_fit} = 0.860$ which is clearly less than unity.
% This result indicates that the corresponding dynamo is not in the dipolar dominated regime.
% }

%{\color{red}
%Fig.~\ref{fig:fig_11} shows the magnetic energy density at the CMB and surface for two cases; (a) $Ra/Ra_{\rm crit} = 8.0$ at $r_i/r_o = 0.15$ and (b) $Ra/Ra_{\rm crit} = 11.9$ at $r_i/r_o = 0.15$. 
%In case (a), the dipolarity is $f_{\rm dip} = 0.494 > 0.35$ and the ratio of extrapolation from the fitting is $f_{\rm mag\_fit} = 1.435 > 1$. 
%These values mean a dipolar component is dominant. 
%This is consistent with the spectra of the CMB and surface. 
%In case (b), the dipolarity is $f_{\rm dip} = 0.117 < 0.35$ and the ratio of extrapolation from the fitting is $f_{\rm mag\_fit} = 0.322 < 1$. 
%These values mean a dipolar component is not dominant. 
%This is consistent with the spectra of the CMB and surface, which show a quadrupolar component is dominant. 
%The spectra imply that if paleointensity is large, geomagnetic field was not always dipolar-dominated.
%}

\begin{figure}
\begin{center}
\[
\begin{array}{cc}
\mbox{$E = 10^{-3}$} &
\mbox{$E = 10^{-4}$} \\
\includegraphics*[width=75mm]{Figures/ratio_rac_VS_aspect_Ek3.pdf} &
\includegraphics*[width=75mm]{Figures/ratio_rac_VS_aspect_Ek4.pdf}
\end{array}
\]
\end{center}
\caption{
Dynamo regime in $r_{i} / r_{o} = 0.15$, $0.25$ and $0.35$. Red circles, blue triangles, green squares, and black crosses represent strong dipolar, weak dipolar, non-dipolar, and failed dynamo cases, respectively.
}
\label{fig:dynamo_summary}
\end{figure}

% Considering both $f_{\rm dip}$ and $f_{\rm mag\_fit}$, Fig.~{\color{red}\ref{fig:dynamo_summary}
% } describes the dynamo regime. 
%In Fig.~12, the red circles, blue triangles, green squares, and black crosses represent the strong dipolar, weak dipolar, non-dipolar, and failed dynamo cases, respectively.
% 上記は figure caption に書くべき
{\color{red}
Taking into account $f_{\rm dip}$, $f_{\rm mag\_fit}$, and $E_{\rm mag}/E_{\rm kin}$ in Fig.~\ref{fig:fdip_fit_Eratio}, we show the dynamo regime for respective radial ratios in Fig.~\ref{fig:dynamo_summary}.
% When the magnetic energy is larger/smaller than the kinetic energy in a simulation case, we categorized this as a strong/weak dynamo.
We categorise a dynamo into the strong/weak field regime when the magnetic energy is larger/smaller than the kinetic energy.
}
% Sustaining the dynamo with a smaller inner core size requires a large Rayleigh number. 
% This is consistent with the findings of Heimpel et al.\ \shortcite{Heimpel:2005}. 
%At $r_i/r_o = 0.35$, almost all the sustained dynamo cases were strong dipoles. 
%At $r_i/r_o = 0.25$, there were strong dipolar dynamo cases and weak dipolar dynamo cases. 
%At $r_i/r_o = 0.15$, there were weak dipolar and non-dipolar dynamo cases.
{\color{red}
As seen in Figs~\ref{fig:Snap_non_dipoler_E1-3}--\ref{fig:Snap_Dipoler_E1-4}, the flow and temperature in the case of $r_i / r_o = 0.15$ are azimuthally localised, and their spherical harmonic order $m$, corresponding to their azimuthal wave number, is smaller than those in the case of $r_i / r_o = 0.25$ and $0.35$.
In short, the number of columnar convection cells is a few or so for $r_i / r_o = 0.15$.
This suggests that the region where the axial component of magnetic field is efficiently generated is comparatively small.
Therefore, to sustain the magnetic field due to dynamo action, strong convective motions are required; that is, a large magnetic Reynolds number is required.
In this respect, Heimpel {\it et al.} \shortcite{Heimpel:2005} mentioned that the axial dipole magnetic field can be sustained even by a single pair of convective columns at $E = 3.0 \times 10^{-4}$ and $Pm = 5$.
It should be noted that a larger magnetic Prandtl number adopted in a dynamo simulation can lead to a larger magnetic Reynolds number as given by eq.~(\ref{eq:Rm}).
As listed in Table~6, the magnetic field cannot be maintained by dynamo action at $Ra / Ra_{\rm crit} \le 7.103$ and $E = 10^{-4}$ for $r_i / r_o = 0.15$.
In these cases, only one pair of convection columns is found to be stably occurred.
On the other hand, in the case of $Ra / Ra_{\rm crit} = 8.523$, which is the lowest Rayleigh number to sustain the magnetic field, the number of pairs of convection columns is not temporally fixed at one but varies between one and two.
Thus, the number of columnar convective cells in relation to the inner core size is likely to influence the dynamo regime.

As mentioned above, Heimpel {\it et al.} \shortcite{Heimpel:2005} pointed out that there is a large difference between $R_d$ for $r_i / r_o = 0.15$ and for $r_i / r_o = 0.25$.
Lhuillier {\it et al.} \shortcite{Lhuillier:2019} found that a transition with respect to frequencies of polarity reversal from a small inner-core regime to a large inner-core regime occurs between $r_i / r_o = 0.20$ and $r_i / r_o = 0.22$ at $E^* = \nu / \Omega r_o^2 = 2.75 \times 10^{-3}$.
Thus, the ratio of the inner core to the outer core radii, $r_i / r_o \sim 0.2$ seems to be a key factor in the temporal evolution of geodynamo.
This point will be discussed later.

The present results suggest that the range of the magnetic Reynolds number, $Rm$, to sustain the axial dipole magnetic field is narrower for the smaller inner core size.
Consequently, the range of the Rayleigh number, $Ra$, for self-sustained dynamo is also narrower for the small radius ratio.
}

\begin{figure*}
\begin{center}
\[
\begin{array}{cc}
 {\rm (a)}~~~~~ \mbox{$E = 10^{-3}$, $Pm = 5.0$} &
 {\rm (b)}~~~~~ \mbox{$E = 10^{-3}$, $Pm = 5.0$} \\
 \mbox{$Ra / Ra_{\rm crit} = 3.056$, $r_{i} / r_{o} = 0.25$} &
 \mbox{$Ra / Ra_{\rm crit} = 15.60$, $r_{i} / r_{o} = 0.15$} \\
\includegraphics*[width=75mm]{Figures/A25_Ra220_forces.png} &
 \includegraphics*[width=75mm]{Figures/sph_shell_401_forces.png} \\
 {\rm (c)}~~~~~ \mbox{$E = 10^{-4}$, $Pm = 2.0$} &
 {\rm (d)}~~~~~ \mbox{$E = 10^{-4}$, $Pm = 2.0$} \\
 \mbox{$Ra / Ra_{\rm crit} = 7.916$, $r_{i} / r_{o} = 0.25$} &
 \mbox{$Ra / Ra_{\rm crit} = 56.82$, $r_{i} / r_{o} = 0.15$} \\
 \includegraphics*[width=75mm]{Figures/sph_shell_387_forces.png} &
 \includegraphics*[width=75mm]{Figures/sph_shell_376_forces.png}
\end{array}
%\includegraphics*[width=160mm]{Figures/Fig6.png}
\]
\end{center}
\caption{Power spectrum of the forces averaged over the  convective layer ($r_{i}+0.05 < r < r_{o} - 0.05$) as a function of spherical harmonic degree $l$. Results are averaged over the 0.5 times of the magnetic diffusion time and the standard deviations of the power are shown by shaded areas.}
\label{fig:force_balance_spectr}
\end{figure*}

{\color{red}
In recent numerical simulations of geodynamo, very low Ekman numbers have been adopted for approach to Earth's core conditions (e.g., Aubert 2019; Schaeffer {\it et al.} 2017).
In the present study, we adopted $E = 10^{-3}$ and $E = 10^{-4}$,
which have also been adopted in many numerical simulations of geodynamo \cite{pena:2018}.
% In these dynamos, since the Ekman number is $E = 1 \times 10^{-3}$, thermal convection in a rotating spherical shell is not the rapidly rotating regime reported by Gastine {\it et al.}  \shortcite{Gastine:2016} and Long {\it et al.} \shortcite{Long:2020}. 
According to Gastine {\it et al.}  \shortcite{Gastine:2016} and Long {\it et al.} \shortcite{Long:2020}, however, thermal convection in a rotating spherical shell for $E = 10^{-3}$ would not be in the rapidly rotating regime.
Therefore, we here calculate the dynamic Elsasser number, $\Lambda_d$, defined by Soderlund {\it et al.} \shortcite{Soderlund:2012} for the numerical dynamos in the present study to evaluate the relative importance of the Lorentz to Coriolis forces. 
As listed in Tables~\ref{table:Summary_15}--8, $\Lambda_d$ is found to be approximately 0.01 to 0.5, which means that the Coriolis force is significantly % large 
{\color{red} dominant} to form a columnar convection structure in the rotating fluid spherical shell.
% Therefore, Coriolis force is sufficient to form a columnar convection structure in the dynamos.
}

{\color{blue}
We also investigate the force balance in the convective region ($r_{i} + 0.05 < r < r_{o} - 0.05$) in the strong dipolar and multipolar dominated regimes.
Fig.~\ref{fig:force_balance_spectr} shows power spectra of forces in eq.~(\ref{eq:momentum}) with respect to the sperical harmonic degree, $l$.
% As seen in the force balance in the case with multipolar solution in $E = 10^{-3}$ case, the geostrophic balance is still dominant in this case, but the force balance is departing from the geostrophic balane. 
As seen in Fig.~\ref{fig:force_balance_spectr}(b), corresponding to a multipolar solution at $E = 10^{-3}$, the geostrophic balance is clearly dominant.
% The significant difference between the dipolar and multipolar cases is also seen in the amplitude of the Lorentz force and inertia. 
The significant difference between the force balances for the dipolar and multipolar regimes is also seen in the amplitude of the Lorentz force and inertia.
% In the multipolar cases, the amplitude of inertia is larger than the Lorentz force in the dominant scale, while the Lorentz force is larger than the inertia in the all length scales in the cases with dipolar solutions. 
In the multipolar dynamo regime, the amplitude of inertia is larger than that of the Lorentz force in large length scales (lower degrees of spherical harmonics), while the amplitude of the Lorentz force is larger than that of the inertia in the all length scales in the dipolar dynamo regime.
}
{\color{green}
The result suggests that the amplitudes of inertia and Lorentz force control the upper bound of $Rm$ to sustain the dipolar magnetic field.
}% 何の上限かわかりません。Rm?

{\color{blue}
% 以下は前の方に移動・修正した。
% On the other hand, the dominant length scale might be take into account to explain the reason why larger magnetic Reynolds number is required. As seen in Figs~\ref{fig:Snap_non_dipoler_E1-3} to \ref{fig:Snap_Dipoler_E1-4}, the flow and temperature patterns in the  $r_i/r_o  = 0.15$ cases are more localized and have smaller zonal wave numbers ({\it i.e.} fewer numbers of upwellong flow) than that in the $r_i/r_o  = 0.25$ and 0.35 cases. In the qualitative expression, more strong convection in each column is required to sustain the axial dipolar field if the convection is characterized by the fewer number of localized columns.
}

\subsection{Implications for temporal evolution of geodynamo}

{\color{red}
As mentioned above, the Ekman number, $E$, adopted in the present study is found to be low enough that the Coriolis force is dominant, although the Ekman number for the real Earth is much lower.
Keeping this in mind, we consider temporal evolution of geodynamo as implied by the results obtained so far.
% The dipolarity at the CMB of the present Earth is $f_{\rm dip} = 0.64$, which is calculated from the 12th IGRF model \cite{Thebault:2015}. 
}

{\color{red}
% The dipolarity of the present geomagnetic field is $f_{\rm dip} = 0.63$ as calculated from the 13th International Geomagnetic Reference Field model \cite{Alken:2021}.
The axial dipole component of the present geomagnetic field is known to be dominant.
The 13th International Geomagnetic Reference Field model \cite{Alken:2021} leads to the value of dipolarity, $f_{\rm dip} = 0.63$, and another index, $f_{\rm mag\_fit} = 4.79$.
% Lmax = 12 で計算
These values clearly indicate the mentioned above.
The former value is consistent with the values of $f_{\rm dip}$ obtained for $r_i / r_o = 0.35$ in this study, while the latter value is much smaller than those of $f_{\rm mag\_fit}$.
This may result from the morphology of the magnetic field, in which the equatorially antisymmetric component is dominant.
}

% The present radius ratio, $r_i/r_o$, is 0.35. 
% The range of the dipolarity calculated from results of our numerical simulations of geodynamo for $r_i/r_o  = 0.25$ and $0.35$ covers the present Earth's dipolarity. 
% The morphology of the sustained magnetic field in both ratios is an Earth-like field. 
% The ratio of the extrapolation from fitting in the present Earth is $f_{\rm mag\_fit} = 4.79$. 
% Here, $f_{\rm mag\_fit}$ is larger than approximately half of the present Earth's value for almost all the cases at $r_i/r_o = 0.35$, while $f_{\rm mag\_fit}$ is smaller than that of almost all the cases at $r_i/r_o = 0.25$. 
% Dipole dominance at $r_i/r_o = 0.35$ is slightly less than that of the present Earth. 
% More magnetic energy, i.e., $l > 2$, is distributed at $r_i/r_o = 0.25$ than the present Earth. 
% In contrast, the dipolarity at $r_i/r_o = 0.15$ is smaller than the present Earth's dipolarity in all cases. 
% The dipole component is not dominant.

% In numerical dynamos at $r_i/r_o = 0.35$, we verified that the transition between the dipole and multipole is $f_{\rm dip} \approx 0.35$ \cite{Uli:2006,Olson:2011}.
% Our results are consistent with this transition. 
% While dipolarity is an effective index if dynamos can be categorized into large and small dipolarity groups, the combination of dipolarity and the ratio of extrapolation from fitting assesses the dipolar dominance if the dipolarity changes gradually, as in our results.
% 上記の段落は 4.1 の内容 & すでに述べられている

%At $r_i/r_o =0.15$, an axial dipole formed by a single column . 
%In this study, a dipole also formed by some azimuthally localized narrow columns around the dynamo-onset cases.
%Here, $E_{\rm mag}$ is always smaller than $E_{\rm kin}$ in all Ra cases. 
%The magnitude relationship is the same as that of Heimpel et al.\ \shortcite{Heimpel:2005}. 

% \subsection{Comparison with previous studies and implication to the past Earth's dynamo}
{\color{blue}
% 以下は前の方に移動・修正した。
% Heimpel {\it et al.} \shortcite{Heimpel:2005} has been represented that an axial dipole formed by a single set of columns. 
% In the present study, a dipole also formed by some azimuthally localized narrow columns around the dynamo-onset cases in $E = 1.0 \times 10^{-3}$ regime. 
% However, As seen in the top panel of Fig.~\ref{fig:Snap_Dipoler_E1-4}, two convection columns can be observed in the dipolar-dominant case in the $E = 1.0 \times 10^{-4}$ regime. 
% At the $Ra/Rac = 8.523$ case, which is the lowest $Ra$ case to sustain the magnetic field, the dominant number of the columns changes between one set and two sets frequently. 
% Stable one set of convection columns is observed in the failed dynamo cases with $Ra/Rac = 3.551$ and 7.103. 
% It may be possible to sustain the dynamo with one set of convection with larger magnetic Prandtl number such as $Pm = 5$ because Heimpel {\it et al.} \shortcite{Heimpel:2005} chose $E = 3.0 \times 10^{-4}$ and $Pm = 5.0$. 
% In the present study, $E_{\rm mag}$ is larger than $E_{\rm kin}$ in $E = 1.0 \times 10^{-4}$ regime, while $E_{\rm mag}$ is smaller than $E_{\rm kin}$ in the $E = 1.0 \times 10^{-3}$ regime and results by Heimpel {\it et al.} \shortcite{Heimpel:2005}. 
% The results suggest that the required magnetic Reynolds number $Rm$ to sustained the dynamo has to be larger than the transition from the dipolar and multi-polar dynamo regime in the large Ekman number cases.
}

{\color{red}
As discussed above, a larger magnetic Reynolds number is required to sustain a magnetic field by dynamo action for a smaller inner core, whereas a too large magnetic Reynolds number causes a weak-field dynamo in the multipolar regime.
Such a dynamo cannot reproduce characteristics of the geomagnetic field, and it is not satisfied with any criteria for palaeomagnetic field behaviour as suggested by Sprain {\it et al.} \shortcite{Sprain:2019}.
This means that the range of the magnetic Reynolds number, in which a dipolar dominated dynamo is sustained for $r_i / r_o = 0.15$, is narrower than those for $r_i / r_o = 0.25$ and $0.35$.
In general, the larger the Rayleigh number is, the larger the magnetic Reynolds number is.
Therefore, the result also implies that the range of the Rayleigh number for a self-sustained dynamo is also narrower for $r_i / r_o = 0.15$.
Here, it should be noted that we normalised the length scale by the thickness of a spherical shell, $L = r_o - r_i = r_o (1 - r_i / r_o)$; that is, the length scale depends on the spherical shell radius ratio, $r_i / r_o$, as $r_o = 1 / (1 - r_i / r_o) = 1.53846$ for $r_i / r_o = 0.35$ corresponding to the ratio of the inner to the outer core radii for the present Earth.
After formation of the core, its radius would not have changed so much, whereas the inner core size would have been growing after its nucleation.
Therefore, we adopt the Rayleigh number for $r_i / r_o = 0.35$,
%
\begin{equation}
    Ra^* = \frac{\alpha g_o \Delta T D^3}{\kappa\nu}
    ~~~{\rm with}~~~D = r_o (1 - r_i / r_o)
    ~~~{\rm for}~~~r_o = 1.53846,
\label{eq:Ra_0.35}
\end{equation}
%
as a measure of $Ra$.
Using $r_o = 1.17647$ for $r_i / r_o = 0.15$ and $r_o = 1.33333$ for $r_i / r_o = 0.25$, we obtain the Rayleigh numbers as
%
\begin{equation}
\begin{array}{l}
    \displaystyle
    Ra^* \left( \frac{1.53846}{1.17647} \right)^3
    = 2.23623 \times Ra^* 
    ~~~{\rm for} ~~~ r_i / r_o = 0.15,\\
    \displaystyle
    Ra^* \left( \frac{1.53846}{1.33333} \right)^3
    = 1.53619 \times Ra^* 
    ~~~{\rm for} ~~~ r_i / r_o = 0.25.
\end{array}
\end{equation}
%
These values clearly signify that a much larger Rayleigh number is necessary to generate and maintain the magnetic field by dynamo action in a rotating spherical shell with a smaller spherical shell radius ratio.
If dynamo action could generate a dipolar dominated magnetic field at the epoch of $r_i / r_o = 0.15$, the magnetic field would be continuously maintained during the growth of the inner core, because the smallest Rayleigh number for a self-sustained dynamo is lower for a larger inner core as a result of its growth.
The geomagnetic field with a dominant dipole component is found to have been maintained on the basis of the palaeomagnetic study \cite{Merrill:1996}.
Hence, it might be possible to impose a strong constraint on a physical state near the core-mantle boundary in relation to the range of the Rayleigh number for a self-sustained dynamo.
}

{\color{blue}
% The simulation results can be changed by the thermal boundary conditions and source of buoyancy. 
}
{\color{red}
In the present study, we imposed a boundary condition of fixed temperature at the ICB and the CMB.
A difference in thermal boundary conditions and/or source of buoyancy can give rise to different results of numerical geodynamo.
For example, a uniform heat flux condition at the CMB can lead to a larger scale flow which generates a strong magnetic field in the core for low-viscosity geodynamo models \cite{Sakuraba:2009}.
}
% For example, a strong dipole field is sustained with a smaller inner core in the fixed flux calculation \cite{Hori:2010}, changing the core power based on the thermal history \cite{Driscoll:2016}, or the buoyancy gained by light elements \cite{Lhuillier:2019}. 
{\color{red}
In the same way, a strong dipole field can be sustained by dynamo action in a thick outer core ($r_i / r_o = 0.10$) when a fixed heat flux condition is imposed at the CMB \cite{Hori:2010}.
Driscoll \shortcite{Driscoll:2016} demonstrated temporal evolution of a geodynamo model in accordance with imposed conditions varied by a thermal history model.
Lhuillier {\it et al.} \shortcite{Lhuillier:2019} examined the effect of inner-core size on the dipole field as in the present study.
They focused not on the dipolarity but on polarity reversals as behaviour of the dipole field by adopting the so-called compositional convection that drives dynamo action in rotating spherical shells with various ratios of the inner to the outer radii.
It is interesting to note that, as mentioned above, they found a transition of frequencies of polarity reversals at around $r_i / r_o \sim 0.2$ even if a different style of convection is taken into account.
}
{\color{red}
% The present results suggest that the boundary between dipolar and non-dipolar dynamo regime has no $Rm$ dependency on the thermal boundary conditions because the force balance controls the change of dynamo regime. 
The present results suggest that a transition between the dipolar and multipolar dynamo regimes seems to occur at a same magnetic Reynolds number and that it does not depend on the thermal boundary condition, because the transition of dynamo regime is likely to be controlled by the force balance in the outer core.
}
{\color{blue}
% On the other hand, the lower $Rm$ limit to sustain dipolar magnetic field may have a dependency on the thermal conditions because the number of convection columns can depends onthe thermal conditions.
On the other hand, the lower limit of magnetic Reynolds number to sustain a dipolar magnetic field may depend on the thermal boundary conditions, which can be related with the number of convection columns.
}
{\color{red}
% Clarifying how heat flow at boundaries sustains the dipole requires further numerical simulations.
It is necessary to carry out further numerical simulations of geodynamo to clarify how convective motions in rapidly rotating spherical shells driven on heat flux boundary conditions sustain the dipole magnetic field.
}
{\color{blue}
% 以下は前の方に移動・修正した。
% The present results suggest that the range of the magnetic Reynolds number is smaller with the smaller inner core size. 
% Consequently, the Rayleigh number range to sustain the dipolar magnetic field is also smaller with smaller inner core (see Fig.~\ref{fig:dynamo_summary}).
}

% Our proposed method of evaluating the dipolar dominance, $f_{\rm mag\_fit}$, enables {\color{blue} more} quantitative investigations of the magnetic field structure than the dipolarity $f_{\rm dip}$.
% in the past environment.  
% Knowledge from palaeomagnetic analyses, such as VDM and VGP (virtual geomagnetic pole), is acquired on the assumption that the geomagnetic field has been dipolar-dominated in the past \cite{Merrill:1996}. 
% In contrast, the VGP paths and actual behavior of the geomagnetic field are not dipolar-dominant. 
{\color{red}
% We found that although a dipolar component of magnetic field at the surface is large, the dipolar component is not dominant at the CMB in cases of $Ra/Ra_{\rm crit} > 10.1$ with $r_i/r_o = 0.15$ 
{\color{blue}
% in the $E = 1.0 \times 10^{-3}$ regime.
}
% Our results imply that geomagnetic field could be multipolar regardless of strength of paleointensity.
} 
% Investigation of the numerical dynamo with our proposed method is capable of improving the understanding of the actual behavior of the geomagnetic field and paleomagnetic observations.
%
%
% \begin{figure}
% \begin{center}
% \[
% \includegraphics*[width=160mm]{Figures/Fig10_mod.png}
% \]
% \end{center}
% \caption{Magnetic energy density at the CMB of simulation data at $l = 1$ and fitting value of $l = 1$ as a function of spherical harmonic odd degrees in $E = 1.0 \times 10^{-3}$ regieme. (a) Dipolar case at $Ra / Ra_{\rm crit} = 7.1$ at $r_{i} / r_{o} = 0.35$, and (b) Non-dipolar case at $Ra / Ra_{\rm crit} = 10.1$ at $r_{i} / r_{o} = 0.15$.
% }
% \label{fig:mag_fitting}
% \end{figure}
%
%
% \begin{figure}
% \begin{center}
% \[
% \begin{array}{cc}
% \mbox{$E = 1.0 \times 10^{-3}$} &
% \mbox{$E = 1.0 \times 10^{-4}$} \\
% \includegraphics*[width=75mm]{Figures/fdip_vs_Rm_Ek3.pdf} &
% \includegraphics*[width=75mm]{Figures/fdip_vs_Rm_Ek4.pdf} \\
% \includegraphics*[width=75mm]{Figures/ffit_vs_Rm_Ek3.pdf} &
% \includegraphics*[width=75mm]{Figures/ffit_vs_Rm_Ek4.pdf} \\
% \includegraphics*[width=75mm]{Figures/Eratio_vs_Rm_Ek3.pdf} &
% \includegraphics*[width=75mm]{Figures/Eratio_vs_Rm_Ek4.pdf}
% \end{array}
%\includegraphics*[width=80mm]{Figures/Fig9.png}
% \]
% \end{center}
% \caption{The dipolarity $f_{\rm dip}$, the ratio of the dipolar magnetic energy density at the CMB to the extrapolation of the spherical harmonic degree $l = 1$, $f_{\rm mag\_fit}$, and the ratio of the magnetic to kinetic energy densities $E_{\rm mag} / E_{\rm kin}$ averaged over the spherical shells as a function of the magnetic Reynolds numnber $Rm$ for different geometries. $f_{\rm dip}$,  $f_{\rm mag\_fit}$, and $E_{\rm mag} / E_{\rm kin}$ are shown in the top, middle, and bottom panels, respectively. The regimes with $E = 1.0 \times 10^{-3}$ and $10^{4}$ are shown in the left and right panels, respectively. The blue, green, and red points indicate the cases of $r_{i} / r_{o} = 0.15$, 0,25, and 0.35, respectively. 
% }
% \label{fig:fdip_fit_Eratio}
% \end{figure}
%
%
%\begin{figure*}
%\begin{center}
%\[
%\includegraphics*[width=160mm]{Figures/Fig11.png}
%\]
%\end{center}
%\caption{{\color{red} Magnetic energy density at the CMB and surface of simulation data as a function of spherical harmonic degree. (a) Dipolar case at $Ra / Ra_{\rm crit} = 8.0$ at $r_{i} / r_{o} = 0.15$, and (b) Non-dipolar case at $Ra / Ra_{\rm crit} = 11.9$ at $r_{i} / r_{o} = 0.15$.
%}}
%\label{fig:fig_11}
%\end{figure*}
%
%
% \begin{figure}
% \begin{center}
% \[
% \begin{array}{cc}
% \mbox{$E = 1.0 \times 10^{-3}$} &
% \mbox{$E = 1.0 \times 10^{-4}$} \\
% \includegraphics*[width=75mm]{Figures/Eratio_vs_Rm_Ek3.pdf} &
% \includegraphics*[width=75mm]{Figures/Eratio_vs_Rm_Ek4.pdf}
% \end{array}
% \]
% \end{center}
% {\color{red}
% \caption{
% {\color{red}
% The ratio of the magnetic to kinetic energy densities averaged over the spherical shells as a function of the magnetic Reynolds numnber $Rm$ for different geometries. The blue, green, and red points indicate the cases of $r_{i} / r_{o} = 0.15$, 0,25, and 0.35, respectively.
% }}
% }
% \label{fig:Eratio_vs_Rm}
% \end{figure}
%
%
% \begin{figure*}
% \begin{center}
% \[
% \begin{array}{cc}
%  \mbox{$E = 1.0 \times 10^{-3}$, $Pm = 5.0$} &
%  \mbox{$E = 1.0 \times 10^{-3}$, $Pm = 5.0$} \\
%  \mbox{$Ra / Ra_{\rm crit} = 3.056$, $r_{i} / r_{o} = 0.25$} &
%  \mbox{$Ra / Ra_{\rm crit} = 15.60$, $r_{i} / r_{o} = 0.15$} \\
% \includegraphics*[width=75mm]{Figures/A25_Ra220_forces.png} &
%  \includegraphics*[width=75mm]{Figures/sph_shell_401_forces.png} \\
%  \mbox{$E = 1.0 \times 10^{-4}$, $Pm = 2.0$} &
%  \mbox{$E = 1.0 \times 10^{-4}$, $Pm = 2.0$} \\
%  \mbox{$Ra / Ra_{\rm crit} = 7.916$, $r_{i} / r_{o} = 0.25$} &
%  \mbox{$Ra / Ra_{\rm crit} = 56.82$, $r_{i} / r_{o} = 0.15$} \\
%  \includegraphics*[width=75mm]{Figures/sph_shell_387_forces.png} &
%  \includegraphics*[width=75mm]{Figures/sph_shell_376_forces.png}
% \end{array}
% %\includegraphics*[width=160mm]{Figures/Fig6.png}
% \]
% \end{center}
% \caption{Power spectrum of the forces averaged over the  convective layer ($r_{i}+0.05 < r < r_{o} - 0.05$) as a function of spherical harmonic degree $l$. Results are averaged over the 0.5 times of the magnetic diffusion time and the standard deviations of the power are shown by shaded area.}
% \label{fig:force_balance_spectr}
% \end{figure*}
%
%
%
% \begin{figure}
% \begin{center}
% \[
% \begin{array}{cc}
% \mbox{$E = 1.0 \times 10^{-3}$} &
% \mbox{$E = 1.0 \times 10^{-4}$} \\
% \includegraphics*[width=75mm]{Figures/ratio_rac_VS_aspect_Ek3.pdf} &
% \includegraphics*[width=75mm]{Figures/ratio_rac_VS_aspect_Ek4.pdf}
% \end{array}
% \]
% \end{center}
% \caption{
% Dynamo regime in $r_{i} / r_{o} = 0.15$, 0.25, and 0.35. Red circles, blue triangles, green squares, and black crosses represent strong dipolar, weak dipolar, non-dipolar, and failed dynamo cases, respectively.
% }
% \label{fig:dynamo_summary}
% \end{figure}
%

%
\section{CONCLUSIONS}

{\color{red}
% As the inner core has been growing for approximately one billion years, sustained dynamo conditions with different inner core sizes are required to understand the past Earth's environment. 
The inner core of the Earth has been growing for a very long time after its nucleation due to cooling of the Earth.
% 内核誕生年の例を挙げた方がよい
This means that the spherical shell geometry of the core has been changing continuously.
% To understand the geometry effect, we performed numerical dynamo simulations with three different radius ratios: $r_i/r_o = 0.15$, $0.25$, and $0.35$. 
We performed numerical simulations of geodynamo with three different radius ratios, $r_i / r_o = 0.15$, $0.25$ and $0.35$ to understand the effect of inner-core size on the dipolarity of the geomagnetic field.
%To evaluate the morphology of the magnetic field, especially the dipole component dominancy, we combined two indices: (i) dipolarity, which is widely used for assessing the relative dipole strength in numerical dynamos, and (ii) the ratio of a dipolar extrapolation from the fitting curve of higher degrees; this method is often used to explain Earth’s observed dipolar-dominated field. 
We evaluated the morphology of the magnetic field, especially dipolar dominance, by using two indices; one is $f_{\rm dip}$, the dipolarity widely used to assess the relative strength of the dipole field in numerical dynamos, and the other is $f_{\rm mag\_fit}$ obtained from a curve fitted to a magnetic power spectrum of non-dipole components (only odd degrees of spherical harmonics are used in this study).
% We verified that the ratio of extrapolation from the fitting was valid for determining the dynamo regime. 
We verified that $f_{\rm mag\_fit}$ was valid for determining the dynamo regime.
% Around the transition between the dipolar and non-dipolar regimes ($f_{\rm dip} \approx 0.35$), we assessed the dipolar dominance using the ratio of extrapolation from fitting. 

We found that when a value of $f_{\rm dip}$ is close to the threshold value ($f_{\rm dip} \approx 0.35$) between the dipolar and multipolar regimes, it is better to refer to the other index, $f_{\rm mag\_fit}$, to judge the dynamo regime, and vice versa.
% Based on dipolarity and the ratio of extrapolation from fitting, we limited the magnetic Reynolds number $Rm$ range of the sustained dynamo for all radius ratios. 
We determined the range of the magnetic Reynolds number, $Rm$, which is necessary to sustain dynamos for the three radius ratios on the basis of $f_{\rm dip}$ and $f_{\rm mag\_fit}$.
}
{\color{blue}
The upper limit of $Rm$ to sustain the intense dipole magnetic field does not depend on the radius ratio, since it is controlled by the amplitude ratio between the Lorentz force and advection. 
On the other hand, the lower limit of $Rm$ depends on the inner core size and decreases with increase of the radius ratio, because the number of convection columns increases with increase of the inner core size. 
}
{\color{blue}
Consequently, 
}
%We found that
{\color{red}
% the $Rm$ range the strong dipole became narrower with a smaller inner core size. 
the $Rm$ range to maintain a strong dipole magnetic field is narrower for a smaller inner core.
}

{\color{red}
% While the dependence of dipolarity on the Rayleigh number $Ra$ is similar at $r_i/r_o = 0.25$ and $0.35$, the dipolar dominance becomes weaker with the smaller inner core 
% {\color{blue}
% because the critical Rayleigh number $Ra_{\rm crit}$ increases with the smaller inner core.
% }
The critical Rayleigh number, $Ra_{\rm crit}$ for the onset of thermal convection in a rotating spherical shell is larger for a smaller inner core.
Even though the dependence of dipolarity on the Rayleigh number, $Ra$, for $r_i / r_o = 0.25$ and $0.35$ is similar to each other (Fig.~\ref{fig:fdip_vs_Racratio}), the value of dipolarity is smaller for a smaller inner core.
% There were no strong dipolar dynamos but non-dipolar dynamos at $10.1 < Ra/Ra_{\rm crit} < 15.6$, only at $r_i/r_o = 0.15$ 
For example, any dynamos in the strong dipolar regime were not found at $E = 10^{-3}$ and $10.1 \le Ra / Ra_{\rm crit} \le 15.6$ for $r_i / r_o = 0.15$, 
{\color{blue} 
% in the Ekman number $E = 10^{-3}$ regime 
because the lower limit of $Rm$ to sustain the dynamo becomes larger than the upper limit of $Rm$ to sustain the dipolar magnetic field. 
}
}

{\color{red}
% Our results indicate that changes in the radius ratio largely influence the dynamo regime in numerical dynamos with a fixed temperature boundary condition. 
Our results indicate that the inner core size largely influences the dynamo regime as mentioned by Heimpel {\it et al.} \shortcite{Heimpel:2005} and Lhuillier {\it et al.} \shortcite{Lhuillier:2019}, and that the range of $Ra$ to sustain the dipolar dynamo is narrower for a smaller inner core.
% Further numerical dynamo simulations, by applying different thermal boundary conditions, are required to determine if the intense dipolar magnetic field was sustained in the past Earth with a smaller inner core or entirely without an inner core.
The results were obtained from numerical dynamo models in which a boundary condition of fixed temperature was imposed at the boundary surfaces.
A different thermal boundary condition can give rise to a different result, so that further numerical simulations of geodynamo are required to investigate the range of $Ra$ for self-sustained dynamos by imposing a boundary condition of fixed heat flux.
}


\begin{acknowledgments}
This study used the computational resources of the HPCI system provided by the Research Institute for Information Technology, Kyushu University, and the Cyberscience Center, Tohoku University, through the HPCI System Research Project (Project IDs: hp210022 and hp220040). This research was also supported by the “Computational Joint Research Program (Collaborative Research Project on Computer Science with High-Performance Computing)” at the Institute for Space-Earth Environmental Research, Nagoya University.
\end{acknowledgments}

\vspace{15mm}
{\color{red}
\noindent
{\bf DATA AVAILABILITY}

\noindent
The numerical code, Calypso Ver.~1.2, used in this study is available from the following URL,

\noindent
http://github.com/geodynamics/calyps

The simulation data analysed in this study is available from the corresponding author upon reasonable request.
}
%
\begin{thebibliography}{}
%
{\color{red}
\bibitem[\protect\citename{Alken {\it et al.} }2021] {Alken:2021}
Alken, P., Thebault, E., Beggan, C.D., Amit, H., Aubert, J., Baerenzung, J., Bondar, T.N., Brown, W. J., Califf, S., Chambodut, A., Chulliat, A., Cox, G. A., Finlay, C. C., Fournier, A., Gillet, N., Grayver, A., Hammer, M.D., Holschneider, M., Huder, L., Hulot, G., Jager, T., Kloss, C., Korte, M., Kuang,W., Kuvshinov, A., Langlais, B., Leger, J.-M., Lesur, V., Livermore, P.W., Lowes, F.J., Macmillan, S., Magnes, W., Mandea, M., Marsal, S., Matzka, J., Metman, M. C., Minami, T., Morschhauser, A., Mound, J.E., Nair, M., Nakano, S., Olsen, N., Pavon-Carrasco, F.J., Petrov, V.G., Ropp, G., Rother, M., Sabaka, T.J., Sanchez, S., Saturnino, D., Schnepf, N., R., Shen, X., Stolle, C., Tangborn, A., Toffner-Clausen, L., Toh, H., Torta, J.M., Varner, J., Vervelidou, F., Vigneron, P., Wardinski, I., Wicht, J., Woods, A., Yang, Y., Zeren, Z. \& Zhou, B., 2021. International Geomagnetic Reference Field: the thirteenth generation, {\it Earth Planets Space}, {\bf 73}, 49, doi:10.1186/s40623-020-01288-x.
}
%
\bibitem[\protect\citename{Al-Shamali {\it et al.} }2004] {Al-Shamali:2004}
Al-Shamali, F.M., Heimpel, M.H. \& Aurnou, J.M., 2004. Varying the spherical shell geometry in rotating thermal convection, {\it Geophys.\  Astrophys.\ Fluid Dyn.}, {\bf 98}(2), 153--169, doi:10.1080/0309192041000165928.
%
{\color{red}
\bibitem[\protect\citename{Aubert }2019] {Aubert:2019}
Aubert, J., 2019. Approaching Earth's core conditions in high-resolution geodynamo simulations, \gji {\bf 219}(Supplement 1), S137--S151, doi:10.1093/gji/ggz232.
}
%
\bibitem[\protect\citename{Aubert {\it et al.} }2009] {Aubert:2009}
Aubert, J., Labrosse, S. \& Poitou, C., 2009. Modelling the paleo-evolution of the geodynamo, {\it Geophys.\ J. Int.}, {\bf 179}, 1414--1428, doi:10.1111/j.1365-246X.2009.04361.x.
%
\bibitem[\protect\citename{Biggin {\it et al.} }2015] {Biggin:2015}
Biggin, A.J., Piispa, E.J., Pesonen, L.J., Holme, R., Paterson, G.A., Veikkolainen, T. \& Tauxe, L., 2015. Palaeomagnetic field intensity variations suggest Mesoproterozoic inner-core nucleation, {\it Nature}, {\bf 526}, 245–248.
%
\bibitem[\protect\citename{Christensen {\it et al.} }2001] {Uli:2001}
Christensen, U.R., Aubert, J., Cardin, P., Dormy, E., Gibbon, S., Glatzmaier, G.A., Grote, E., Honkura, Y., Jones, C., Kono, M., Matsushima, M., Sakuraba, A., Takahashi, F., Tilgner, A., Wicht, J. \& Zhnag, K., 2001. A numerical dynamo benchmark, \pepi {\bf 128}, 25--34, doi:10.1016/S0031-9201(01)00275-8.
%
\bibitem[\protect\citename{Christensen \& Aubert  }2006] {Uli:2006}
Christensen, U.R. \& Aubert, J., 2006. Scaling properties of convection-driven dynamos in rotating spherical shells and application to planetary magnetic fields, \gji {\bf 166}(1), 97--144, doi:10.1111/j.1365-246X.2006.03009.x
%
\bibitem[\protect\citename{Driscoll }2016] {Driscoll:2016}
Driscoll, P.E., 2016. Simulating 2 Ga of geodynamo history, \grl {\bf 43}, 5680--5687, doi:10.1002/2016GL068858.
%
{\color{red}
\bibitem[\protect\citename{Gastine {\it et al.}, }2016] {Gastine:2016}
Gastine, T., Wicht, J. \& Aubert, J., 2016. Scaling regimes in spherical shell rotating convection, {\it J. Fluid Mech.}, {\bf 808}, 690--732, doi:10.1017/jfm.2016.659.
}
%
\bibitem[\protect\citename{Heimpel {\it et al.} }2005] {Heimpel:2005}
Heimpel, M.H., Aurnou, J.M., Al-Shamali, F.M. \& Perez N.G., 2005. A numerical study of dynamo action as a function of spherical shell geometry, {\it Earth planet.\ Sci.\ Lett.}, {\bf 236}, 542--557, doi:10.1016/j.epsl.2005.04.032.
%
\bibitem[\protect\citename{Hori {\it et al.} }2010] {Hori:2010}
Hori, K., Wicht, J. \& Christensen U.R., 2010. The effect of thermal boundary conditions on dynamo driven by internal heating, {\it Phys.\ Earth planet.\ Inter.}, {\bf 182}, 85--97, doi:10.1016/j.pepi.2010.06.011.
%
\bibitem[\protect\citename{Kono \& Roberts }2002] {Kono:2002}
Kono, M. \& Roberts, P.H., 2002. Recent geodynamo simulations and observations of the geomagnetic field, {\it Rev.\ Geophys.}, {\bf 40}(4), 1--53, doi:10.1029/2000RG000102.
%
\bibitem[\protect\citename{Kutzner \& Christensen }2002] {Kutzner:2002}
Kutzner, C. \& Christensen, U.R., 2002. From stable dipolar towards reversing numerical dynamos, {\it Phys.\ Earth planet.\ Inter.}, {\bf 131}(1), 29--45, doi:10.1016/S0031-9201(02)00016-X.
%
\bibitem[\protect\citename{Labrosse {\it et al.} }2001] {Labrosse:2001}
Labrosse, S., Poirier, J.-P. \& Le Mou{\"e}l, J.-L., 2001. The age of the inner core, {\it Earth planet.\ Sci.\ Lett.}, {\bf 190}, 111--123.
%
\bibitem[\protect\citename{Langel \& Estes }1982] {Langel:1982}
Langel, R.A. \& Estes, R.H., 1982. A geomagnetic field spectrum, {\it Geophys.\ Res.\ Lett.}, {\bf 9}(4), 250--253.
%
\bibitem[\protect\citename{Lhuillier {\it et al.} }2019] {Lhuillier:2019}
Lhuillier, F., Hulot, G., Gallet, Y. \& Schwaiger, T., 2019. Impact of inner-core size on the dipole field behaviour of numerical dynamo simulations, \gji {\bf 218}, 179--189, doi:10.1093/gji/ggz146.
%
\bibitem[\protect\citename{Long {\it et al.} }2020] {Long:2020}
Long, R.S., Mound, J.E., Davis, C.J. \& Tobias, S.M., 2020. Scaling behavior in spherical shell rotating convection with fixed-flux thermal boundary conditions, {\it J. Fluid Mech.}, {\bf 889}, doi:10.1017/jfm.2020.67.
%
\bibitem[\protect\citename{Lowes }1974] {Lowes:1974}
Lowes, F.J., 1974. Spatial power spectrum of the main geomagnetic field, and extrapolation to the core, {\it Geophys.\ J. Roy.\ astr.\ Soc.}, {\bf 36}(3), 717--730, doi:10.1111/j.1365-246X.1974.tb00622.x.
%
\bibitem[\protect\citename{Matsui {\it et al.} }2014] {Matsui:2014}
Matsui, H., King, E. \& Buffett, B., 2014. Multiscale convection in a geodynamo simulation with uniform heat flux along the outer boundary, {\it Geochem.\ Geophys.\ Geosys.}, {\bf 15},  3212--3225, doi:10.1002/2014GC005432.
%
{\color{red}
\bibitem[\protect\citename{Meduri {\it et al.} }2021] {Meduri:2021}
Meduri, D.G., Biggin, A.J., Davies, C.J., Bono, R.K., Sprain, C.J., \& Wicht, J., 2021. Numerical dynamo simulations reproduce paleomagnetic field behavior, \grl {\bf 48}, e2020GL090544, doi:10.1029/2020GL090544.
}
%
\bibitem[\protect\citename{e.g.\ Merrill {\it et al.} }1996] {Merrill:1996}
Merrill, R.T., McElhinny, M.W. \& McFadden, P.L., 1996. The Magnetic Field of the Earth: Paleomagnetism, the Core, and the Deep Mantle, Academic Press.
%
\bibitem[\protect\citename{Olson {\it et al.} }2011] {Olson:2011}
Olson, P.L., Glatzmaier, G.A. \& Coe, R.S., 2011. Complex polarity reversals in a geodynamo moidel, {\it Earth planet.\ Sci.\ Lett.}, {\bf 304}(1)--(2), 168--179, doi:10.1016/j.epsl.2011.01.031.
%
{\color{red}
\bibitem[\protect\citename{e.g.\ Pe{\~n}a {\it et al.}} 2018]{pena:2018}
Pe{\~n}a, D., Amit, H. \& Pinheiro, K.J., 2018. Deep magnetic field stretching in numerical dynamos, {\it Prog.\ Earth planet.\ Sci.}, {\bf 5}, 8, doi:10.1186/s40645-017-0162-5
}
%
{\color{red}
\bibitem[\protect\citename{Roberts \& Glatzmaier }2001] {RG:2001}
Roberts, P.H.,  Glatzmaier, G.A., 2001. The geodynamo, past, present and future. {\it Geophys.\ astrophys.\ Fluid Dyn.}, {\bf 94}, 47--84.
}
%
{\color{red}
\bibitem[\protect\citename{Schaeffer {\it et al.} }2017] {Schaeffer:2017}
Schaeffer, N., Jault, D., Nataf, H.-C., \& Fournier, A., 2017. Turbulent geodynamo simulations: a leap towards Earth's core. \gji {\bf 211}(1), 1--29, doi:10.1093/gji/ggx265.
}
%
{\color{red}
\bibitem[\protect\citename{Sakuraba \& Roberts }2009] {Sakuraba:2009}
Sakuraba, A \& Roberts, P.H., 2009. Generation of a strong magnetic field using uniform heat flux at the surface of the core. {\it Nature Geosci.}, {\bf 2}, 802--805, doi:10.1038/ngeo0643.
}
%
\bibitem[\protect\citename{Soderlund {\it et al.} }2012] {Soderlund:2012}
Soderlund, K.M., King, E.M. \& Aurnou, J.M., 2012. The influence of magnetic field in planetary dynamo models, {\it Earth planet.\ Sci.\ Lett.}, {\bf 333--334}, 9--20, doi:10.1016/j.epsl.2012.03.038.
%
{\color{red}
\bibitem[\protect\citename{Sprain {\it et al.} }2019] {Sprain:2019}
Sprain, C.J., Biggin, A.J., Davies, C.J., Bono, R.K., \& Meduri, D.G., 2019. An assessment of long duration geodynamo simulations using new paleomagnetic modeling criteria ($Q_{\rm PM}$), \epsl {\bf 526}, 115758, doi:10.1016/j.epsl.2019.115758.
}
%
% \bibitem[\protect\citename{Thebault {\it et al.} }2015] {Thebault:2015}
% Thebault, E., Finlay, C. C., Beggan, C. D., Alken P., Aubert, J., Barrois, O., Bertrand, F., Bondar, T., Boness, A., Brocco, L., Canet, E., Chambodut, A., Chulliat, A., Coisson, P., Civet, F., Du, A., Fournier, A., Fratter, I., Gillet, N., Hamilton, B., Hamoudi, M., Hulot, G., Jager, T., Korte, M., Kuang, W., Lalanne, X., Langlais, B., Leger, J.-M., Lesur V., Lowes, F. J. Macmillan, S., Mandea, M., Manoj, C., Maus, S., Olsen, N., Petrov, V., Ridley, V., Rother, M., Sabaka, T. J., Saturnino, D., Schachtschneider, R., Sirol, O., Tangborn, A., Thomson, A., Toffner-Clausen, L., Vigneron, P., Wardinski, I., \& Zvereva, T., 2015. International Geomagnetic Reference Field: the 12th generation, {\it Earth Planets Space}, {\bf 67}:79, doi:10.1186/s40623-015-0228-9.
%

\end{thebibliography}

%
\end{document}
