% gjilguid2e.tex
% V2.0 released 1998 December 18
% V2.1 released 2003 October 7 -- Gregor Hutton, updated the web address for the style files.

\documentclass{gji}
\usepackage{timet}

\title{Investigation of dipolar dominancy in geodynamo simulations with different inner core sizes}
\author{Y. Nishida$^1$, Y. Katoh$^1$, H. Matsui$^2$, M. Matsushima$^3$, T. Kera$^1$, and A. Kumamoto \\
  %\thanks{Pacific Region Office, GJI} \\
  $^1$ Department of Geophysics, Tohoku University, Sendai, Japan \\
  $^2$ Department of Earth and Planetary Sciences, University of California Davis, Davis, California, USA \\
  $^3$ Department of Earth and Planetary Sciences, Tokyo Institute of Technology, Tokyo, Japan
  }
\date{ }
\pagerange{\pageref{firstpage}--\pageref{lastpage}}
\volume{200}
\pubyear{1998}

%\def\LaTeX{L\kern-.36em\raise.3ex\hbox{{\small A}}\kern-.15em
%    T\kern-.1667em\lower.7ex\hbox{E}\kern-.125emX}
%\def\LATeX{L\kern-.36em\raise.3ex\hbox{{\Large A}}\kern-.15em
%    T\kern-.1667em\lower.7ex\hbox{E}\kern-.125emX}
% Authors with AMS fonts and mssymb.tex can comment out the following
% line to get the correct symbol for Geophysical Journal International.
\let\leqslant=\leq

\newtheorem{theorem}{Theorem}[section]

\begin{document}

\maketitle
%
\begin{summary}
The solid inner core of the Earth has been growing for approximately one billion years due to cooling of the Earth. The changing spherical shell geometry of the Earth’s core is likely to influence on the geodynamo driven by convective motions in the fluid outer core. To understand the geometry effect on the dynamo regime through evolution of the core, we perform numerical simulations of geodynamo with three spherical shell radius ratios: $r_{i}/r_{o} = 0.15$, 0.25, and 0.35, where $r_{i}$ and $r_{o}$ are the inner and outer core radii, respectively. To evaluate the morphology of the magnetic field, we examine two indices about dipole component dominance: (i) $f_{dip}$, dipolarity used to assess the relative strength of the dipole field at the core surface in numerical dynamo models; and (ii) $f_{mag\_fit}$, the ratio of magnetic energy density for the dipole component to that extrapolated from the magnetic power spectrum for the high degree components. We investigate the field morphology estimated from $f_{dip}$, and $f_{mag\_fit}$, and find that $f_{mag\_fit}$ is valid to determine the dynamo regime, even if $f_{dip}$ suggests a transition regime between dipolar and non-dipolar dominance. We also investigate the range of Rayleigh number for sustained dynamos based on $f_{dip}$ and $f_{mag\_fit}$, and find that the range of Rayleigh number for a dynamo characterized by a strong dipole field becomes narrower for a smaller inner core. The $f_{dip}$–dependence on the Rayleigh number for $r_{i}/r_{o} = 0.25$ and 0.35 is similar each other, whereas the $f_{mag\_fit}$–dependence for $r_{i}/r_{o} = 0.35$ is found to be relatively larger than that for $r_{i}/r_{o} = 0.25$. On the other hand, small values of  $f_{dip}$ and  $f_{mag\_fit}$ for $r_{i}/r_{o} = 0.15$ suggest that the dynamo regime is characterized not by a strong dipole field but by non-dipolar dominance. These results indicate that changes in the spherical shell radius ratio largely influence on the dynamo regime in numerical dynamos with a fixed temperature boundary condition.
\end{summary}
%
\section{Introduction}

\end{document}
