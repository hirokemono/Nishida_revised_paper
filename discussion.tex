\section{DISCUSSION}

In numerical dynamos, dipolarity is used for quantification of the magnetic field morphology at the CMB. 
Although dipolarity has been evaluated in some numerical dynamos \cite{Uli:2006,Soderlund:2012}, it is not sufficiently valid in dynamos whose dipolarities are gradually changing (e.g., Aubert {\it et al.} 2009). 
In an observational magnetic field, the dipole is assessed by how far the dipolar component is from the trend of higher degree components \cite{Lowes:1974,Langel:1982}. 
We quantitatively evaluated dipolar component dominance in combination with the dipolarity, comparison of the dipolar magnetic energy, and an extrapolation of $l = 1$ based on the fitting curve of higher degrees.

To quantitatively evaluate the axial dipole component dominancy, we calculated the dipolarity at the CMB, which is defined as follows:
%
\begin{equation}
f_{\rm dip} = 
\left(
\frac{E_{mag}^{(l=1,m=0)} (r=r_o)}
     {\sum_{l=1}^{l_{\rm max}}
      \sum_{m=0}^l E_{\rm mag}^{l,m} (r=r_o)}
\right)^{1/2},
\label{eq:f_dip}
\end{equation}
%
where the magnetic energy at the CMB, $E_{\rm mag} (r=r_o)$, is calculated as follows:
%
\begin{equation}
E_{\rm mag} (r=r_o) = 
  \frac{1}{V_s E Pm} \int_S \frac{1}{2} \bvec{B}^2 dS.
\end{equation}
%
Fig.~7 shows the dipolarity as a function of the Rayleigh number for $r_i/r_o = 0.15$, $0.25$, and $0.35$. 
The dipolarity gradually decreases with an increasing Rayleigh number for $r_i/r_o = 0.25$ and $0.35$. 
The axial dipolar component becomes weak during intense convection. 
The dependency of the dipolarity on the Rayleigh number is similar for the two radius ratio cases, i.e., $r_i/r_o = 0.25$ and $0.35$. 
Here, $f_{\rm dip}$ is always larger than $0.35$ at $r_i/r_o = 0.25$ and $0.35$. 
In contrast, this tendency is different at $r_i/r_o = 0.15$. 
Furthermore, $f_{\rm dip}$ is larger than $0.45$ at $Ra/Ra_{\rm crit} = 8.0$ and $9.0$ while $f_{dip}$ is smaller than $0.35$ at $Ra/Ra_{\rm crit} > 10.1$.

The results of a number of previous numerical dynamo simulations determine the threshold of the dipolar dominance for $f_{\rm dip} = 0.35$. 
To obtain a clearer threshold for the dipole dominance, we focused on the magnetic energy spectrum at the CMB as a function of the spherical harmonic degree, $l$. 
For example, Fig.~8 shows the magnetic energy density as a function of the spherical harmonic degree at $Ra/Ra_{\rm crit} = 2.8$ for $r_i/r_o = 0.25$. 
Using odd-degree components in the magnetic energy from $l = 3$ to $19$, we evaluated the fitting curve as $46.21 \times 1.481^{-l}$. 
At degree $l = 1$, the $E_{\rm mag}$ of the simulation data $E_{\rm mag\_data}^{l = 1}$ was compared with that from the extrapolated value in the fitting function $E_{\rm mag\_fitting}^{l = 1}$. 
Then, we acquired the ratio of $E_{\rm mag}$ for the simulation result to that from the extrapolated value, $E_{\rm mag\_data}^{l = 1}/E_{\rm mag\_fitting}^{l=1}$ (hereafter referred to as $f_{\rm mag\_fit}$. 
We can assess the dipolar component dominance from a higher degree trend based on how much the ratio of the extrapolation from fitting $f_{mag\_fit}$ is larger than $1$. 
Here, $f_{\rm mag\_fit}$ was calculated in all cases and plotted as a function of the Rayleigh number at $r_i/r_o = 0.15$, $0.25$, and $0.35$ (Fig.~9). 
At $r_i/r_o = 0.15$, $f_{\rm mag\_fit}$ was smaller than $1$ at $Ra/Ra_{\rm crit} > 10.1$. 
At $r_i/r_o = 0.25$, $f_{\rm mag\_fit}$ was approximately $2.1$ at $Ra/Ra_{\rm crit} = 2.2$ and gradually decreased to $1.6$ with increase of $Ra/Ra_{\rm crit}$ of up to approximately $8.0$. 
At $r_i/r_o = 0.35$, $f_{\rm mag\_fit}$ was approximately $4.7$ at $Ra/Ra_{\rm crit} = 2.0$, gradually decreasing to $2.1$ with increase of $Ra/Ra_{\rm crit}$ of up to approximately $7.0$.

Comparing the result between $f_{dip}$ and $f_{\rm mag\_fit}$, the common point is that the dipolar dominance decreases with an increasing Rayleigh number. 
At $r_i/r_o = 0.15$, the dipolar non-dominance can be represented by $Ra/Ra_{\rm crit} > 10.1$ for both indices. 
At $r_i/r_o = 0.25$ and $0.35$, the magnitude relationship of $f_{\rm dip}$ and $f_{\rm mag\_fit}$ is reversed at a small $Ra/Ra_{\rm crit}$ value. 
Because the magnetic energy of higher degrees is relatively larger for the total energy at $r_i/r_o = 0.35$ than at $r_i/r_o = 0.25$, $f_{\rm dip}$ is smaller at $r_i/r_o = 0.35$ than at $r_i/r_o = 0.25$. 
Although $E_{\rm mag\_fitting}^{l = 1}$ is almost the same at $r_i/r_o = 0.25$ and $0.35$, $E_{\rm mag\_data}^{l = 1}$ is significantly larger, such that $f_{\rm mag\_fit}$ is larger at $r_i/r_o = 0.35$ than at $r_i/r_o = 0.25$. 
The difference between $f_{\rm dip}$ and $f_{\rm mag\_fit}$ derives from whether a higher-degree spectrum is taken into account or not. 
The dependency of the dipolar dominance on the radius ratio can be revealed by $f_{\rm mag\_fit}$ as it contains information from a higher degree spectrum.

We show an example in which we could not categorize the dipole or non-dipole based only on the dipolarity, i.e., the cases for $f_{\rm dip} = 0.376$ at $Ra/Ra_{\rm crit} = 7.1$ with $r_i/r_o = 0.35$ and $f_{\rm dip} = 0.349$ at $Ra/Ra_{\rm crit} = 10.1$ for $r_i/r_o = 0.15$. 
Fig.~10 shows the CMB spectra for these two cases. 
The dipolar component is dominant against the high degree trend in the former case while it is not dominant in the latter case. 
The ratio of extrapolation from the fitting is $f_{\rm mag\_fit} = 2.071$ in the former case and $f_{\rm mag\_fit} = 0.860$ in the latter case; we observed that the former case is dipolar-dominated while the latter case is non-dipolar dominated. 
The results obtained for $f_{\rm mag\_fit}$  also indicate the dependence of the dipolar dominance on the inner core size. 
The dipolar dominance becomes weaker with a smaller inner core by calculating the dipolar magnetic energy at the CMB \cite{Heimpel:2005}. 
In this study, although this tendency was not observed from the dipolarity, it was clear based on the ratio of extrapolation from fitting.

{\color{red}
Fig.~11 shows the magnetic energy density at the CMB and surface for two cases; (a) $Ra/Ra_{\rm crit} = 8.0$ at $r_i/r_o = 0.15$ and (b) $Ra/Ra_{\rm crit} = 11.9$ at $r_i/r_o = 0.15$. 
In case (a), the dipolarity is $f_{\rm dip} = 0.494 > 0.35$ and the ratio of extrapolation from the fitting is $f_{\rm mag\_fit} = 1.435 > 1$. 
These values mean a dipolar component is dominant. 
This is consistent with the spectra of the CMB and surface. 
In case (b), the dipolarity is $f_{\rm dip} = 0.117 < 0.35$ and the ratio of extrapolation from the fitting is $f_{\rm mag\_fit} = 0.322 < 1$. 
These values mean a dipolar component is not dominant. 
This is consistent with the spectra of the CMB and surface, which show a quadrupolar component is dominant. 
The spectra imply that if paleointensity is large, geomagnetic field was not always dipolar-dominated.}

Considering both $f_{\rm dip}$ and ${f_{\rm mag\_fit}$, Fig.~{\color{red}12} describes the dynamo regime. 
%In Fig.~12, the red circles, blue triangles, green squares, and black crosses represent the strong dipolar, weak dipolar, non-dipolar, and failed dynamo cases, respectively.
% 上記は figure caption に書くべき
When the magnetic energy is larger/smaller than the kinetic energy in a simulation case, we categorized this as a strong/weak dynamo. 
Sustaining the dynamo with a smaller inner core size requires a large Rayleigh number. 
This is consistent with the findings of Heimpel et al.\ \shortcite{Heimpel:2005}. 
At $r_i/r_o = 0.35$, almost all the sustained dynamo cases were strong dipoles. 
At $r_i/r_o = 0.25$, there were strong dipolar dynamo cases and weak dipolar dynamo cases. 
At $r_i/r_o = 0.15$, there were weak dipolar and non-dipolar dynamo cases.

{\color{red}
In these dynamos, since the Ekman number is $E = 1 \times 10^{-3}$, thermal convection in a rotating spherical shell is not the rapidly rotating regime reported by Gastine et al.\  \shortcite{Gastine:2016} or Long et al.\ \shortcite{Long:2020}. 
We calculated the dynamic Elsasser number ($\Lambda_d$) defined by Soderlund et al.\ \shortcite{Soderlund:2012} in these dynamos to evaluate the relative strength of the Lorentz to Coriolis forces. 
It is found that $\Lambda_d$ is approximately 0.01 to 0.1. 
Therefore, Coriolis force is sufficient to form a columnar convection structure in the dynamos.
}

The dipolarity at the CMB of the present Earth is $f_{\rm dip} = 0.64$, which is calculated from the 12th IGRF model \cite{Thebault:2015}. 
The present radius ratio, $r_i/r_o$, is 0.35. 
The range of the dipolarity calculated from results of our numerical simulations of geodynamo for $r_i/r_o  = 0.25$ and $0.35$ covers the present Earth's dipolarity. 
The morphology of the sustained magnetic field in both ratios is an Earth-like field. 
The ratio of the extrapolation from fitting in the present Earth is $f_{\rm mag\_fit} = 4.97$. 
Here, $f_{\rm mag\_fit}$ is larger than approximately half of the present Earth's value for almost all the cases at $r_i/r_o = 0.35$, while $f_{\rm mag\_fit}$ is smaller than that of almost all the cases at $r_i/r_o = 0.25$. 
Dipole dominance at $r_i/r_o = 0.35$ is slightly less than that of the present Earth. 
More magnetic energy, i.e., $l > 2$, is distributed at $r_i/r_o = 0.25$ than the present Earth. 
In contrast, the dipolarity at $r_i/r_o = 0.15$ is smaller than the present Earth's dipolarity in all cases. 
The dipole component is not dominant.

In numerical dynamos at $r_i/r_o = 0.35$, we verified that the transition between the dipole and non-dipole is $f_{\rm dip} \approx 0.35$ \cite{Uli:2006,Olson:2011}.
Our results are consistent with this transition. 
While dipolarity is an effective index if dynamos can be categorized into large and small dipolarity groups, the combination of dipolarity and the ratio of extrapolation from fitting assesses the dipolar dominance if the dipolarity changes gradually, as in our results.

At $r_i/r_o =0.15$, an axial dipole formed by a single column \cite{Heimpel:2005}. 
In this study, a dipole also formed by some azimuthally localized narrow columns around the dynamo-onset cases. 
Here, $E_{\rm mag}$ is always smaller than $E_{rm kin}$ in all Ra cases. 
The magnitude relationship is the same as that of Heimpel et al.\ \shortcite{Heimpel:2005}. 
A strong dipole is sustained with a smaller inner core in the fixed flux calculation \cite{Hori:2010}, changing the core power based on the thermal history \cite{Driscoll:2016}, or the buoyancy gained by light elements \cite{(Lhuillier:2019}.
Clarifying how heat flow at boundaries sustains the dipole requires further numerical simulations.

Our proposed method of evaluating the dipolar dominance, $f_{\rm mag\_fit}$, enables quantitative investigations of the magnetic field structure in the past environment. 
Knowledge from paleomagnetic analyses, such as VDM and VGP (virtual geomagnetic pole), is acquired based on the assumption that the geomagnetic field was dipolar-dominated in the past \cite{Merrill:1996}. 
In contrast, the VGP paths and actual behavior of the geomagnetic field are not dipolar-dominant. 
{\color{red}
We found that although a dipolar component of magnetic field at the surface is large, the dipolar component is not dominant at the CMB in cases of $Ra/Ra_{\rm crit} > 10.1$ with $r_i/r_o = 0.15$. 
Our results imply that geomagnetic field could be non-dipolar regardless of strength of paleointensity.} 
Investigation of the numerical dynamo with our proposed method is capable of improving the understanding of the actual behavior of the geomagnetic field and paleomagnetic observations.
